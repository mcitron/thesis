%% Title
\titlepage[Imperial College London\\Department of Physics]%
{A dissertation submitted to Imperial College London\\
  for the degree of Doctor of Philosophy}

\clearpage
\textbf{The copyright of this thesis rests with the author and is made available under a Creative Commons 
Attribution Non-Commercial No Derivatives licence. Researchers are free to copy, distribute or 
transmit the thesis on the condition that they attribute it, that they do not use it for commercial 
purposes and that they do not alter, transform or build upon it. For any reuse or redistribution, 
researchers must make clear to others the licence terms of this work.}
%\clearpage


%% Abstract
\begin{abstract}%[\smaller \thetitle\\ \vspace*{1cm} \smaller {\theauthor}]
  %\thispagestyle{empty}
The discovery of the Standard Model (SM) Higgs boson is one of the primary physics objectives of the Large Hadron Collider at CERN. 
This thesis describes a search carried out for the SM Higgs boson on data collected during the 2011 and 2012 proton-proton (pp) collision 
runs with the CMS detector corresponding to integrated luminosities of $5.1 fb^{-1}$ and $5.3 fb^{-1}$ respectively.
A detailed description of the search for the SM Higgs boson decaying to two photons 
from the full dataset collected at CMS during the 2011 pp collision run is provided. 
In particular, the development of signal and background modelling techniques used for statistical interpretations of the data are highlighted.
Results of the search using these techniques from the 2011 dataset are presented. 
In addition, an update to the analysis including data taken during 2012 is described and the 
results from the combined 2011 and 2012 analyses given.
Results from the combination of several Higgs decay channels at CMS are reported, including those presented in
the International Conference on High Energy Physics in July 2012 at which the announcement of discovery was made.
Ongoing studies to ascertain the properties of the new particle are discussed and preliminary results from the combined 
7 and 8 TeV datasets (corresponding to $5.1fb^{-1}$ and $12.2fb^{-1}$ respectively) are presented.
\end{abstract}


%% Declaration
\begin{declaration}
I, the author of this thesis, hereby declare the work contained in this 
document to be my own. 
Studies conducted and results produced by the author are indicated in the 
main body of text.
All figures labelled ``CMS'' are sourced directly from CMS publications, 
including those produced by the author and have, been referenced as such 
in the figure caption. Where the figure is sourced from a CMS document which 
is unpublished or from a preliminary public document (marked ``CMS Preliminary''), 
a reference to that document is included.
All figures and studies taken from external sources are referenced appropriately 
throughout this document.
%  \vspace*{1cm}
  \begin{flushright}
    Nicholas Wardle
  \end{flushright}
\end{declaration}


%% Acknowledgements
\begin{acknowledgements}
  I would like to thank foremost my parents, Pat and David, who have provided me every opportunity to 
pursue research in Physics. Their unwaivering support and encouragement has been
an endless source of determination throughout my education. In addition, I would like to thank my friends and colleagues (they know who they are) who provided much needed distraction from study and helping me appreciate other aspects of life in Geneva and London.  
  Secondly, I thank my supervisors Jonathan Hays and Gavin Davies for guiding me through my PhD research.
The mix of enthusiasm for the subtleties of data analysis techniques and expertise in maintaining the 
bigger picture have provided many hours of educational and entertaining discussion.
  I'd like to thank the Imperial College $\Hgg$ group and the CMS $\Hgg$ group for providing a platform 
to discuss ideas and results in an open and often welcoming manner.  
  Finally, I would like to thank the STFC for providing the funding for my research and 
 in particular allowing for the time spent in Geneva.

\end{acknowledgements}


%% Preface
%\begin{preface}
% Insert preface here...
%\end{preface}

%% ToC
\tableofcontents

\listoffigures
\listoftables
%% Strictly optional!
\frontquote%
  {Un bon mot ne prouve rien.}%
  {Fran\c{c}ois-Marie Arouet (Voltaire)}
