\chapter{Statistical results and interpretation}
\section{Likelihood model}
\label{sec:likelihood}

Consider a given category of event as defined by \njet, \nb~and \scalht, which are in the following identified with \htcat. 
In each category, the signal is extracted using the discriminating variable \mht. 
Templates of the \mht~distribution are built for the signal and the background processes 
using the simulated samples described in Sec.?? The \mht binning is tabulated in Sec.~\ref{sec:mhtBinning}.
The binning of the templates is chosen taking into account both the limited statistics in the simulation and 
the control of the background in data. 
A maximum statistical uncertainty from the simulated events of 50\% (corresponding to four unweighted events) 
is required in each bin of the template in order to ensure a statistically meaningful prediction. 
A minimum bin width constraint of 50 GeV is applied in order to reduce the bin-by-bin migration 
due to the finite \mht resolution.

The likelihood model in each bin in \mht~and \htcat~is split into hadronic and control components linked
by floating parameters for the prediction. In one \htcat~category, j, the hadronic component may be written as,

\begin{multline}
\label{eq:hadronicLikelihood}
\mathcal{L}^{j}_{\mathrm{had}} = \prod_i \mathrm{Pois}(n^{j,i}_{\mathrm{had}} |\, b^{j,i}_{\zInv~,had}\times\phi^{j}(\mu\rightarrow\zInv~)\times a^{j}\times\rho^{j,i}_{\zInv~,had}\, + \\ 
b^{j,i}_{\ttW,\mathrm{had}}\times\phi^{j}(\mu\rightarrow\ttW)\times a^{j}\times\rho^{j,i}_{\ttW,had}\, + \,s^{j,i}_{\mathrm{had}}\times\rho^{j,i}_{s,had})
\end{multline}

where $b^{j}$ are the predicted number of events from simulation, the $a^{j}$ parameter is unconstrained and 
fully correlated with the prediction of the signal region may be connected 
to the control region (see below), while the $\phi^{j,i}$ contain the systematic uncertainties on the 
transfer factors on the relevant prediction from the data driven tests described in Section ?? and the 
$\rho^{j}$ contain the systematics from variations in simulation as well as the uncertainty from the limited number of 
simulated events for background and signal (described in Section ?? and ??). The constraint terms
for these systematic uncertainties are detailed in Section ??.

The component of the likelihood for the \mj~control region, which is not categorised in \mht~, can be written as

\begin{equation}
\label{eq:muLikelihood}
\mathcal{L}^{j}_{\mathrm{\mu}} = \mathrm{Pois}(n^{j}_{\mathrm{\mu}} |\, b^{j}_{\mu}\times a^{j}\times\rho^{j}_{\mu} + \,s^{j}_{\mathrm{\mu}}\times\rho^{j}_{s,\mu})
\end{equation}

similarly to Equation, $\rho^{j}_{\mu}$ contains the uncertainty in the \htcat from variations in simulation. The $s^{j}_{\mathrm{\mu}}$
econdes the signal \emph{contamination} in the control region which is small by design. The connection between the control and signal region
is encoded by the unconstrained $a^{j}$ parameter. The \zInv~component in the signal 
region is also predicted using the \gj and \mmj regions. By rewriting $\phi^{j}(\mu\rightarrow\zInv~)\times a^{j}$ and $\phi^{j}(\mu\rightarrow\ttW)\times a^{j}$
as ${a'}(\mu\rightarrow\zInv~)$ and $a'(\mu\rightarrow\ttW~)$ respectively, the connections to the \gj~and \mmj~regions
may be written as, 

\begin{align}
\label{eq:mumuLikelihood}
\mathcal{L}^{j}_{\mathrm{\mu\mu}} &= \mathrm{Pois}(n^{j}_{\mathrm{\mu\mu}} |\, b^{j}_{\mu\mu}\times \left({a'}^{j}/\phi^{j}(\mu\mu\rightarrow\zInv~)\right)\times\rho^{j}_{\mu\mu} + \,s^{j}_{\mathrm{\mu\mu}}\times\rho^{j}_{s,\mu\mu}) \\
\mathcal{L}^{j}_{\mathrm{\gamma}} &= \mathrm{Pois}(n^{j}_{\mathrm{\gamma}} |\, b^{j}_{\gamma}\times \left({a'}^{j}/\phi^{j}(\gamma\rightarrow\zInv~)\right)\times\rho^{j}_{\gamma} + \,s^{j}_{\mathrm{\gamma}}\times\rho^{j}_{s,\gamma})
\end{align}

where parameters are defined as in Equations~\ref{eq:hadronicLikelihood} and~\ref{eq:muLikelihood}. 
The $\phi^{j}$ appear on the denominator as the connection between the control and signal region is 
inverted.

The modifier and constraint terms of the systematic uncertainties depend on the source
and can be summarised as,
\begin{itemize}
\item The transfer factor systematics (in $\phi$) are taken to be \emph{log normal} uncertainties
such that for an uncertainty, k, the likelihood includes a modifier term on
the connection between the relevant control and signal processes of $(1+k)^{\theta}$ and a constraint term for $\theta$ which 
is Gaussian (normal) with mean 0 and width 1~\cite{templateMorphing}. These uncertainties 
are correlated per topology and \scalht bin (pair of \scalht bin for uncertainties derived using \mmj)
\item The systematic uncertainties from variations in simulation (in $\rho$) are included using vertical template morphing.
The yields are interpolated quadratically between the $\pm 1\sigma$ variations for each source of
uncertainty and extrapolated linearly beyond this range~\cite{templateMorphing}. The constraint term is again Gaussian
with mean 0 and width 1. These uncertainties are fully correlated across all categories.
\item The poisson uncertainty due to the limited number of simulated events is approximated using
two parameters per bin which multply the total background and signal contributions and which 
are Gaussian constrained. These uncertainties are fully uncorrelated across all categories.
\end{itemize}

The total likelihood can be written as a product over all \htcat~bins,

\begin{equation}
\mathcal{L} = \prod_{j\in\htcat} \mathcal{L}^{j}_{\mathrm{had}} \times \mathcal{L}^{j}_{\mathrm{\mu\mu}} \times \mathcal{L}^{j}_{\mathrm{\gamma}} \times \mathcal{L}^{j}_{\mathrm{\mu}}
\end{equation}

The background predictions for the signal region $b^{j}_{\mathrm{had}}$ are extracted from a likelihood fit 
using the control regions in Eq.~\ref{eq:controlLikelihood} without the signal region (the signal region is \emph{masked}). 
This control region only fit provides the best knowledge of the background yields in the signal region, 
and the predictions can therefore be used to re-interpret the analysis. The values of $b_j$ are 
scaled in the control and signal regions by the control region only fit values of $a_j$.
This allows the expected limits to reflect the results of the control region only fit.
In the second step, the fit is done using the full likelihood (Eq.~\ref{eq:total_likelihood}) and all 
the correlations between the backgrounds in control regions and signal region are taken into account.

\section{Procedure for deriving limits}
