\chapter{Statistical results and interpretation}
\section{Likelihood model}
\label{sec:likelihood}

Consider a given category of event as defined by \njet, \nb~and \scalht, which are in the following identified with \htcat. 
In each category, the signal is extracted using the discriminating variable \mht. 
Templates of the \mht~distribution are built for the signal and the background processes 
using the simulated samples described in Sec.?? The \mht binning is tabulated in Sec.~\ref{sec:mhtBinning}.
The binning of the templates is chosen taking into account both the limited statistics in the simulation and 
the control of the background in data. 
A maximum statistical uncertainty from the simulated events of 50\% (corresponding to four unweighted events) 
is required in each bin of the template in order to ensure a statistically meaningful prediction. 
A minimum bin width constraint of 50 GeV is applied in order to reduce the bin-by-bin migration 
due to the finite \mht resolution.

The likelihood model in each bin in \mht~and \htcat~is split into hadronic and control components linked
by floating parameters for the prediction. In one \htcat~category, j, the hadronic component may be written as,

\begin{multline}
\label{eq:hadronicLikelihood}
\mathcal{L}^{j}_{\mathrm{had}} = \prod_i \mathrm{Pois}(n^{j,i}_{\mathrm{had}} |\, b^{j,i}_{\zInv~,had}\times\phi^{j}(\mu\rightarrow\zInv~)\times a^{j}\times\rho^{j,i}_{\zInv~,had}\, + \\ 
b^{j,i}_{\ttW,\mathrm{had}}\times\phi^{j}(\mu\rightarrow\ttW)\times a^{j}\times\rho^{j,i}_{\ttW,had}\, + b^{j,i}_{\text{QCD},had}\times\omega^{j,i}_{\text{QCD},had}\\\
+ \,r\times s^{j,i}_{\mathrm{had}}\times\rho^{j,i}_{s,had}) 
\end{multline}

where $b^{j}$ are the predicted number of events from simulation for the electroweak
backgrounds and from the method described in~\ref{} for the QCD multijet component ($b^{j,i}_{\text{QCD},had}$), 
the $a^{j}$ parameter is unconstrained and fully correlated with the prediction of the signal region may be connected 
to the control region (see below), while the $\phi^{j,i}$ contain the systematic uncertainties on the 
transfer factors on the relevant prediction from the data driven tests described in Section ?? and the 
$\rho^{j}$ contain the systematics from variations in simulation as well as the uncertainty from the limited number of 
simulated events for background and signal (described in Section ?? and ??), r is an uncontrained parameter 
termed the \emph{signal strength}, and $\omega_{\text{QCD}}$ contains the uncertainties on the QCD multijet component. 
The constraint terms for the systematic uncertainties are detailed in Section ??.

The component of the likelihood for the \mj~control region, which is not categorised in \mht, can be written as

\begin{equation}
\label{eq:muLikelihood}
\mathcal{L}^{j}_{\mathrm{\mu}} = \mathrm{Pois}(n^{j}_{\mathrm{\mu}} |\, b^{j}_{\mu}\times a^{j}\times\rho^{j}_{\mu} + \,r \times s^{j}_{\mathrm{\mu}}\times\rho^{j}_{s,\mu})
\end{equation}

similarly to Equation, $\rho^{j}_{\mu}$ contains the uncertainty in the \htcat from variations in simulation. The $s^{j}_{\mathrm{\mu}}$
econdes the signal \emph{contamination} in the control region which is small by design. The connection between the control and signal region
is encoded by the unconstrained $a^{j}$ parameter. The \zInv~component in the signal 
region is also predicted using the \gj and \mmj regions. By rewriting $\phi^{j}(\mu\rightarrow\zInv~)\times a^{j}$ and $\phi^{j}(\mu\rightarrow\ttW)\times a^{j}$
as ${a'}(\mu\rightarrow\zInv~)$ and $a'(\mu\rightarrow\ttW~)$ respectively, the connections to the \gj~and \mmj~regions
may be written as, 

\begin{align}
\label{eq:mumuLikelihood}
\mathcal{L}^{j}_{\mathrm{\mu\mu}} &= \mathrm{Pois}(n^{j}_{\mathrm{\mu\mu}} |\, b^{j}_{\mu\mu}\times 
\left({a'}^{j}/\phi^{j}(\mu\mu\rightarrow\zInv~)\right)\times\rho^{j}_{\mu\mu} + \,r\times s^{j}_{\mathrm{\mu\mu}}\times\rho^{j}_{s,\mu\mu}) \\
\mathcal{L}^{j}_{\mathrm{\gamma}} &= \mathrm{Pois}(n^{j}_{\mathrm{\gamma}} |\, b^{j}_{\gamma}\times 
\left({a'}^{j}/\phi^{j}(\gamma\rightarrow\zInv~)\right)\times\rho^{j}_{\gamma} + \,r\times s^{j}_{\mathrm{\gamma}}\times\rho^{j}_{s,\gamma})
\end{align}

where parameters are defined as in Equations~\ref{eq:hadronicLikelihood} and~\ref{eq:muLikelihood}. 
The $\phi^{j}$ appear on the denominator as the connection between the control and signal region is 
inverted.

The modifier and constraint terms of the systematic uncertainties depend on the source
and can be summarised as,
\begin{itemize}
\item The transfer factor systematics (in $\phi$) and uncertinaties on the QCD multijet contribution (in $\omega$)  
are taken to be \emph{log normal} uncertainties such that the logarithm of the variable has 
a Gaussian (normal) constraint~\cite{templateMorphing}. These uncertainties are correlated per topology and \scalht bin 
(pair of \scalht bin for uncertainties derived using \mmj)
\item The systematic uncertainties from variations in simulation (in $\rho$) are included using vertical template morphing.
The yields are interpolated quadratically between the $\pm 1\sigma$ variations for each source of
uncertainty and extrapolated linearly beyond this range~\cite{templateMorphing}. The constraint term is Gaussian
with mean 0 and width 1. These uncertainties are fully correlated across all categories.
\item The poisson uncertainty due to the limited number of simulated events is approximated using
two parameters per bin which multply the total background and signal contributions and which 
are Gaussian constrained. These uncertainties are fully uncorrelated across all categories.
\end{itemize}

The total likelihood can be written as a product over all \htcat~bins,

\begin{equation}
\label{eq:totalLikelihood}
\mathcal{L} = \prod_{j\in\htcat} \mathcal{L}^{j}_{\mathrm{had}} \times \mathcal{L}^{j}_{\mathrm{\mu\mu}} 
\times \mathcal{L}^{j}_{\mathrm{\gamma}} \times \mathcal{L}^{j}_{\mathrm{\mu}}
\end{equation}

The background predictions for the signal region $b^{j}_{\mathrm{had}}$ are extracted from a likelihood fit 
using the control regions in Eq.~\ref{eq:totalLikelihood} without the signal region (the signal region is \emph{masked}). 
This control region only fit provides the best knowledge of the background yields in the signal region, 
and the predictions can therefore be used to re-interpret the analysis. The values of $b_j$ are 
scaled in the control and signal regions by the control region only fit values of $a_j$.
This allows the expected limits (see Section~\ref{sec:limits}) to reflect the results of the control region only fit.
In the second step, the fit is done using the full likelihood (Eq.~\ref{eq:total_likelihood}) and all 
the correlations between the backgrounds in control regions and signal region are taken into account.

\section{Procedure for deriving limits}
\label{sec:limits}
The results of the \alphat~search are interpreted by deriving upper limits on the signal strength
at 95\% confidence level for simplified models. This section describes the procedure for deriving
these upper limits. A more comprehensive treatment can be found in~\cite{asymp}.

In the following section the signal strength is considered as the parameter of interest (POI)
and all other parameters in the fit are termed the \emph{nuisance parameters} ($\boldsymbol{\theta}$). Considering
the likelihood defined in~\ref{eq:totalLikelihood}, the profile likelihood ratio can be defined as

\begin{equation}
\label{eq:profile}
\lambda(r) = \frac{\mathcal{L}(r,\hat{\boldsymbol{\theta}}(r))}{\mathcal{L}(\hat{r},\hat{\boldsymbol{\theta}})},
\end{equation}

where $\hat{\boldsymbol{\theta}}$ and $\hat{r}$ are the values of $\boldsymbol{\theta}$ and r that maximise $\mathcal{L}$
, maximum-likelihood (ML) estimators, while $\hat{\boldsymbol{\theta}}(r)$ is the value
of $\boldsymbol{\theta}$ that maximises $\mathcal{L}$ for the specified value of r. 

Given that $r < 0$ is unphysical the profile likelihood is modified as
\begin{equation}
\label{eq:profileNew}
\tilde{\lambda}(r) = 
\begin{cases}
\lambda(0)\quad r \le 0, \\ 
\lambda(r)\quad r > 0. \\ 
\end{cases}
\end{equation}

The test statistic used to derive the upper limit on r is then defined as

\begin{equation}
t_r = 
\begin{cases}
-2\,\text{ln}\,\tilde{\lambda(r)}\quad &\hat{r} \le r, \\ 
0 \quad &\hat{r} > r. \\ 
\end{cases}
\end{equation}

Considering Eq.~\ref{eq:profile}, $t_r$ will be zero at the ML value of r.
Increasing values of $t_r$ represent less compatibility of that value of r with
the observed data. The test statistic is set to zero for $\hat{r} > r$ as
values of r less than $\hat{r}$ are not part of the \emph{rejection region} for upper limits. 

The probability density function, $f(t_r|r)$, for $t_r$ may be built by 
generating pseudo datasets or approximated using the asymptotic formulae 
detailed in~\cite{asymp}. The \alphat~analysis uses the asymptotic
approximation for $f(t_r|r)$. The p-value, $p_r$, can then be defined as

\begin{equation}
p_r = \int_{t_{r,obs}}^{\infty}\, f(t_r|r)\, dt_r
\end{equation}

where ${t_{r,obs}}$ is the observed value of the test statistic. Finally,
the $\text{CLs}$ is defined as

\begin{equation}
\text{CLs}(r) = \frac{p_r}{1-p_0}.
\end{equation}

The upper limit on r is defined as the value of $r$ which corresponds to 
$\text{CLs}(r) = 0.05$ ($r_{95}$). Values of r greater than this are said to be excluded at 95\%
confidence level.

The expected limit on a given signal strength, $r_{95}^{exp}$, is defined by the median value of the distribution
of $r_{95}$ built from pseudo datasets generated with no signal contribution (r = 0). The variation in the 
expected limit may also be estimated by the relevant quantiles from this distributuon.
For the \alphat~analysis, the value and variations of $r_{95}^{exp}$ are approximated using 
the asymptotic formulae detailed in~\cite{asymp}.
