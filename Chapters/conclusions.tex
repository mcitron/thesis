\chapter{Conclusions}

The \alphat search for SUSY has been presented using a 12.9 \ifb 
dataset of 13 \TeV~p-p collisions collected by the CMS experiment. 
The search uses a hadronic final state containing jets and 
significant~\met to achieve sensitivity to a wide range of SUSY
models. Dedicated variables, including \alphat and \bdphi, are used 
to mitigate backgrounds and residual backgrounds 
are robustly estimated using simulation and control regions in data.
The results of the \alphat search following the strategy
described in this thesis has been published using 
a 2.3 \ifb dataset collected in 2015~\cite{alphat2015} while the results 
of a search using the full 36 \ifb dataset collected by CMS in 2016 
is being prepared for submission.

Earlier incarnations of the \alphat~analysis have been carried out
on datasets of 8\TeV~p-p collisions. The analysis has been
fully updated and extended for the results presented in this thesis. 
In particular, the addition of the \mht dimension has robustly improved the sensitivity of the search. 
No evidence for BSM physics is observed and the results are 
therefore interpreted using simplified SUSY models. As shown in Chapter~\ref{cha:statisticalResults},
the regions of parameter space excluded for both gluino mediated 
and direct bottom and top squark production have been 
significantly extended.

The \alphat search is sensitive to models not considered in the
interpretations presented in this thesis. Such models include 
complete models of SUSY and generic models of dark matter. 
A procedure for facilitating the re-interpretation of the results of the search to evaluate the 
impact on such models has been presented. This procedure 
may be applied to a wide range of searches for BSM physics~\cite{simp-lik}.

The L1 trigger jet and energy sum algorithms have been
developed to take advantage of the upgraded L1 trigger hardware.
These provide improved performance in jet identification, 
reconstruction and pileup mitigation. The upgraded algorithms have
been in use for the full 2016 dataset and have been exploited
by the \alphat analysis in designing inclusive signal region selections.

No evidence has been found for BSM physics in the \alphat analysis 
or any search at CMS. The improvements to the analysis and substantial
increase in centre of mass energy has allowed significant gains in
the masses excluded from the end of Run 1. The LHC will continue 
to provide data up to a luminosity of $\sim 3000 \ifb$ over the 
remainder of its lifetime. The best opportunity for discovery of 
BSM physics is likely to come through exploiting this dataset 
using sophisticated analysis techniques as well as targeted searches 
for new types of signatures.

