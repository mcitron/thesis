\chapter{Introduction}
\label{cha:introduction}

In the 20th century, great advances have been made in the
understanding of the properties and interactions of 
the fundamental particles that make up our universe. 
The Standard Model (SM) of particle physics provides an 
extraordinarily successful description of the phenomena 
that have been observed at a plethora of experiments.

Despite these successes, the SM is known to be incomplete.
From a theoretical perspective, the theory cannot provide a description of gravitational
interactions on a microscopic scale. In addition, multiple astronomical observations 
have supported the existence of a weakly interacting particle (Dark Matter) that
comprises around five times more of the mass in the universe than particles 
predicted in the SM.

Supersymmetry (SUSY) is an additional symmetry between bosonic and fermionic particles 
that is among the best motivated of theories that can resolve these
and other issues of the SM. The features of the SM and SUSY models are detailed 
in Chapter~\ref{cha:theory}. Most supersymmetric theories contain a stable,
weakly interacting lightest supersymmetric particle (LSP) which has the properties 
of a dark matter candidate.

The Large Hadron Collider (LHC) is a proton-proton collider designed to probe the 
SM with collisions at the highest ever centre of mass energy (13~\TeV).
The Compact Muon Solenoid (CMS) detector is among two general purpose detectors
designed to provide precise measurements of the energy, position and momenta of
the particles produced in these collisions. The LHC and CMS have run successfully at a lower centre of 
mass energy (7-8~\TeV), discovering the Higgs boson and providing the strongest 
constraints on a range of SUSY models. The LHC and CMS are described in Chapter~\ref{cha:detector}
while the reconstruction of the data collected by CMS is detailed in Chapter~\ref{cha:reco}.

The Level-1 trigger system is a crucial component of the CMS detector, responsible 
for making the initial decision on whether an event should be recorded 
for full analysis. As the data taking conditions become more extreme, the hardware
and algorithms used by this system must be fully upgraded. The jet finding and energy sum algorithms 
are particularly important for SUSY analyses. The Level-1 Trigger upgrade is described in 
Chapter~\ref{cha:triggerUpgrade}. 

The \alphat analysis is a search for SUSY using data recorded by the CMS detector. 
Sensitivity to beyond SM physics requires strong rejection of backgrounds while
maintaining acceptance to signal. Experimental signatures of SUSY include significant
hadronic activity in the form of jets and significant~\emph{missing energy} due to the production
of the LSP. Chapter~\ref{cha:alphat} describes the strategy of the \alphat search, which uses
dimensionless variables to mitigate the otherwise dominant background containing fake missing energy.
This Chapter also described how the \alphat~search aims to achieve inclusive sensitivity 
to a wide range of models by finely categorising events according to 
their energy, missing energy, number of jets and number of b tagged jets.

In order to extract a possible signal contribution the background components passing selection 
and associated systematic uncertainties must be robustly determined. In Chapter~\ref{cha:backgroundPrediction}
the procedures for determining backgrounds containing true and fake missing
energy are detailed. The systematic uncertainties are determined using tests with 
both data and simulation. 

The compatibility of the data with the SM hypothesis is assessed using a maximum likelihood fit 
as detailed in Chapter~\ref{cha:statisticalResults}. As no evidence is observed for BSM physics,
the results are interpreted using simplified supersymmetric models which evaluate the reach
of the search for different SUSY event topologies. 

Finally, searches cannot interpret their results in the plethora of models to which 
they may be sensitive. A procedure is given to facilitate the re-interpretation
of the \alphat~search, such that its impact on any BSM physics may be approximated.
This procedure is applicable to many searches and is described fully in 
Chapter~\ref{cha:simplifiedLikelihood}.
