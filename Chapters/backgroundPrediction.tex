\chapter{Background estimation for the \alphat search}

The accurate estimation of SM background yields is crucial for a sensitive and robust
search for new physics. The $\alphat$ analysis uses data driven estimations for both the
QCD multijet and electroweak background components of the backgrounds. As described below, 
these are determined using the dedicated QCD background estimation and transfer factor 
method techniques respectively. The reliance on the modelling of the backgrounds 
is therefore reduced through the use of these control regions. The predictions for
each \nj,\nb,\scalht bin is discussed in this section. 

The use of data control regions mitigates the effect of mismodelling in simulation, however,
differences between control and signal region selection mean residual biases may remain.
The simulated events are therefore reweighted to account for known discrepancies, 
such as the modelling of pileup. The corrections and the corresponding systematic uncertainties
are described in Section ??. Additional data driven validations of the predictions from the
control regions are carried out using \emph{Closure tests} between the control regions. 
These are discussed in Section ?? and used to derive additional uncertainties to cover effects 
that may not have been included in the systematic uncertainties derived using simulation.

The modelling of the \mht~variable is taken directly from simulation in each \nj,\nb,\scalht bin.
The validation of this modelling and derivation of related systematic uncertanties using
data in the control regions is described in Section ??.

\section{Datasets and simulated samples}
\subsection{Datasets}
\subsection{Simulated samples}
\section{Corrections to simulation}
\subsection{MC reweighting}
\subsection{Studies in data}
\section{Background estimation}
\subsection{Electroweak background prediction}
The electroweak backgrounds are predicted using the \emph{transfer factor} (TF)
method. The control regions are binned identically
to the signal region in \scalht, \nj and \nb. The simulation and data observed in the control region,
$\nobs^{\rm control}(\njet,\nb,\scalht)$ and $\nsim^{\rm control}(\njet,\nb,\scalht)$ 
, as well as the simulation in the signal region, $\nsim^{\rm signal}(\njet,\nb,\scalht)$, 
is used to predict the bkacground in the signal region, $\npre^{\rm signal}(\njet,\nb,\scalht)$. 
The TF is defined using the ratio of the number of events predicted in 
simulation in the signal region and control region for the relevant processes,

\begin{equation}
  \label{equ:tf-ratio}
  {\rm TF} = \frac{N_{\rm MC}^{\rm signal}(\njet,\nb,\scalht)}{N_{\rm
      MC}^{\rm control}(\njet,\nb,\scalht)}.
\end{equation}

The prediction of the relevant process in the signal region can then be written as

\begin{equation}
  \label{equ:pred-method}
  \npre^{\rm signal}(\njet,\nb,\scalht) = \frac{N_{\rm MC}^{\rm
      signal}(\njet,\nb,\scalht)}{N_{\rm MC}^{\rm
      control}(\njet,\nb,\scalht)} \times \nobs^{\rm
    control}(\njet,\nb,\scalht).
\end{equation}

The \mj control region is used to predict the \wj and \ttbar backgrounds while the
\znunu background is predicted using the \mmj and \gj control regions. As described
in Section ?? the selections in the control region closely resemble those made for
the signal region. This ensures that kinematically similar processes and objects
are chosen in the control and signal regions.

Many systematic effects are expected to cancel in the transfer factor. 
% An example
% of this can be seen in figure ?? where the variation expected from simulation in the 
% signal region using simulation is compared to the variation in the transfer factor
% prediction for ??. 
% For the purposes of the transfer factor prediction the signal region is split into two background sources,
% named \zInv and \ttW. 

\subsection{QCD background prediction}
\section{Systematic uncertainties}
\subsection{Variation method}
\subsection{Closure test method}
\section{Addition of the \mht dimension}
