\chapter{Background estimation for the \alphat search}

The accurate estimation of SM background yields is crucial for a sensitive and robust
search for new physics. The $\alphat$ analysis uses data driven estimations for both the
QCD multijet and electroweak background components of the backgrounds. As described below, 
these are determined using the dedicated QCD background estimation and transfer factor 
method techniques respectively. The reliance on the modelling of the backgrounds 
is therefore reduced through the use of these control regions. The predictions for
each \nj,\nb,\scalht bin is discussed in this section. 

The use of data control regions mitigates the effect of mismodelling in simulation, however,
differences between control and signal region selection mean residual biases may remain.
The simulated events are therefore reweighted to account for known discrepancies, 
such as the modelling of pileup. The corrections and the corresponding systematic uncertainties
are described in Section ??. Additional data driven validations of the predictions from the
control regions are carried out using \emph{Closure tests} between the control regions. 
These are discussed in Section ?? and used to derive additional uncertainties to cover effects 
that may not have been included in the systematic uncertainties derived using simulation.

The modelling of the \mht~variable is taken directly from simulation in each \nj,\nb,\scalht bin.
The validation of this modelling and derivation of related systematic uncertanties using
data in the control regions is described in Section ??.

\section{Datasets and simulated samples}
The analysis detailed in this thesis uses datasets collected during the 25ns running of the 
LHC at 13 \TeV~during 2016 as well as simulated samples to model the backgrounds and
signal contributions.
\subsection{Data}
The events recorded by CMS are collected and categorised depending on the trigger selections, 
detailed in Section ??. For the signal region and hadronic control region, data passing the \alphat-\scalht,
\mht-\met and \scalht triggers are collected into the HTMHT, MET and JetHT \emph{primary datasets}.
For the other control regions, data passing the single photon and single muon triggers are collected 
into the SinglePhoton and SingleMuon primary datasets. These primary datasets are filtered to remove
any overlaps. The total integrated luminosity of each dataset is measured as XX\ifb.
\subsection{Simulated samples}
Simulated samples are necessary for background and signal model prediction. As detailed below, 
different \emph{generators} are used to produce the processes expected to contribute to
the \alphat analysis.

The processes with the highest contributions to the signal and control regions, 
including \zj, Drell-Yan (\dy) + jets, \gj, \ttbar, \wj and QCD multijet events, are generated at leading order (LO) 
using \MADGRAPH \AMCATNLO~\cite{Alwall:2014hca}. The same generator is used at next-to-leading order (NLO)
to generate samples of s-channel production of single top and \ttV events.
The t-channel and tW-channel single top samples are generated using \POWHEG~\cite{Alioli:2010xd} at NLO.
The diboson samples, WW, WZ and ZZ, are generated at NLO using \PYTHIA~\cite{PYTHIA}. 
The simulated background samples are normalised using cross sections calculated with NLO and NNLO precision.%CITE?
The full detector response is simulated using the \GEANTfour package for these samples.

The signal samples include gluino-mediated and direct pair production of squarks, in
association with up to two additional partons. These are generated using \MADGRAPH \AMCATNLO
with the sparticle decay, taking 100\% branching fraction to the specified final state, 
performed using \PYTHIA. The cross sections are calculated with
NLO plus next-to-leading-logatithm (NLL) accuracy~\cite{sparticleXs}. The detector response
is simulated using the CMS fast simulation package~\cite{fastsim}.

The XX and YY parton distribution functions (PDFs) are used for the generators described above
The showering and hadronisation is performed using \PYTHIA~\cite{PYTHIA}, with subsequent 
$\tau$ lepton decay simulated using the dedicated \TAUOLA generator~\cite{TAUOLO}. The effects
of pileup are simulated by combining the generated output with a minimum bias sample 
before the detector simulation.

\section{Corrections to simulation}
\subsection{MC reweighting}
\subsection{Studies in data}
\section{Background estimation}
\subsection{Electroweak background prediction}
The electroweak backgrounds are predicted using the \emph{transfer factor} (TF)
method. The control regions are binned identically
to the signal region in \scalht, \nj and \nb. The simulation and data observed in the control region,
$\nobs^{\rm control}(\njet,\nb,\scalht)$ and $\nsim^{\rm control}(\njet,\nb,\scalht)$ 
, as well as the simulation in the signal region, $\nsim^{\rm signal}(\njet,\nb,\scalht)$, 
is used to predict the background in the signal region, $\npre^{\rm signal}(\njet,\nb,\scalht)$. 
The TF is defined using the ratio of the number of events predicted in 
simulation in the signal region and control region for the relevant processes,

\begin{equation}
  \label{equ:tf-ratio}
  {\rm TF} = \frac{N_{\rm MC}^{\rm signal}(\njet,\nb,\scalht)}{N_{\rm
      MC}^{\rm control}(\njet,\nb,\scalht)}.
\end{equation}

The prediction of the relevant process in the signal region can then be written as

\begin{equation}
  \label{equ:pred-method}
  \npre^{\rm signal}(\njet,\nb,\scalht) = \frac{N_{\rm MC}^{\rm
      signal}(\njet,\nb,\scalht)}{N_{\rm MC}^{\rm
      control}(\njet,\nb,\scalht)} \times \nobs^{\rm
    control}(\njet,\nb,\scalht).
\end{equation}

The \mj control region is used to predict the \wj and \ttbar backgrounds while the
\znunu background is predicted using the \mmj and \gj control regions. The use of 
the transfer factor method within the likelihood used for interpreting the results of the
\alphat~analysis is described in detail in Section ??.

The selections in the control region closely resemble those made for
the signal region. This ensures that similar objects and kinematic processes
are chosen in the control and signal regions. The selection on the \alphat and
\bdphi variables are removed for the control regions to increase the statistical 
power of the control regions while selections on quantities such as the invariant
or transverse mass ensure the \mj and \mmj control regions are enriched in
W, Z and \ttbar as appropriate. These selections also ensure the control regions
are signal depleted.

The transfer factors account for differences in the cross sections and branching fractions,
acceptance and reconstruction efficiencies and kinematic requirements between signal 
and control regions. The values of the transfer factors for the prediction from the \mj,
\mmj and \gj control regions are shown in figure ??.

Many systematic effects in the modelling of \scalht, \nb and \njet for the relevant processes 
are expected to cancel or be mitigated in the transfer factor. Residual uncertainties from
sources such as the potential mismodelling of kinematics, theoretical uncertainties (for example
in predicting \znunu using \gj events), and mismodelling of the reconstruction of the objects
used in the control region. The determination of these uncertainties using both simulation 
and tests in data are discussed in detail in Section~\ref{sec:syst-uncs}. 
% An example
% of this can be seen in figure ?? where the variation expected from simulation in the 
% signal region using simulation is compared to the variation in the transfer factor
% prediction for ??. 
% For the purposes of the transfer factor prediction the signal region is split into two background sources,
% named \zInv and \ttW. 

\subsection{QCD background prediction}
\section{Systematic uncertainties in the transfer factors}
\label{sec:syst-uncs}
This section describes the determination of systematic uncertainties 
in the transfer factors though both variations of simulation
and using the data driven \emph{closure tests}. 

\subsection{Variation method}
In section ?? the corrections applied to the simulated samples are summarised. For each 
simulated event the correction may apply to the weight carried by that events (such as b tag efficiency
or pileup reweighting) or to the property of objects in the event (such as the jet energy scale). 
These corrections carry uncertainties that may effect the predictions of the \alphat~analysis.
To determine their effect, for each source of uncertainty the correction is varied to their
value at $\pm$ 1 $\sigma$ and the prediction from simulation re-evaluated. These uncertainties are
propagated to the likelihood as described in section ??. In this section the systematic variations
of the transfer factor for each effect is discussed. The corresponding figures referenced in
this section are in ??.

\subsection{Closure test method}
\section{Addition of the \mht dimension}
