\chapter{Introduction and theory}

In this chapter the current best theory of particle physics, the standard model (SM),
is outlined. Outstanding problems are detailed and used to motivate beyond
standard model (BSM) physics and in particular Supersymmetry (SUSY). 
The features of supersymmetric theories are outlined and the possible 
experimental signatures which would enable such a theory to be 
discoverd are detailed.

Natural units, Einstein summation convention
and Feynman slash notation are used throughout. Electric charges 
are in units of the charge of the electron.

\label{chap:theory}
\section{The standard model}

\label{sec:sm}
The standard model (SM) of particle physics is a quantum field theory (QFT) whose excitations
correspond to the elementary particles forming all known matter and the mediators of
all known forces~\cite{ftsm}. The SM contains spinor fields that result in the spin-1/2 fermions
vector fields resulting in spin-1 bosons, and a scalar field that results in a spin-0
boson. 

The spin-1/2 fermions compose the known matter in the universe and are summarised 
in Table~\cite{tab:fermions}. These are split into \emph{quarks} which carry colour charge (see Section~\ref{sm-gs})
and leptons which do not. Each of the three generations of quarks and leptons vary only in
mass from the other generations in their quantum numbers.

The spin-1 fermions mediate the forces in the SM and are summarised in Table~\ref{tab:bosons}. 
The gluon and photon are massless while the W and Z bosons are massive. 
The mechanism for determining the masses of these particles in discussed in 
Section~\ref{sec:sm-gs} and gives rise to a further fundamental spin-0 particle, the Higgs boson.

The SM contains two classes of symmetries:

\begin{itemize}
\item Space-time symmetries corresponding to translations and rotations of the space-time coordinates.
The SM satisfies the Poincar\'{e} group of space-time transformations that define special relativity. 
\item \emph{Gauge} symmetries corresponding to transformations of the fields within the SM.
\end{itemize}

The mechanism by which the guage symmetries give rise to the properties and interactions 
of the particles in the SM is detailed in the remainder of this section. 

\begin{table}
  \caption[The fundamental spin-1/2 fermions observed in nature separated into their three generations. 
  Each particle shown also has an antiparticle with opposite charge and identical mass]{The fundamental fermions observed in nature separated into their three generations. 
  Each particle shown also has an antiparticle with opposite charge and identical mass~\cite{pdg}.}
  \label{tab:fermions}
  \begin{tabular}{ccccccc}
  \hline\hline
  &\multicolumn{3}{|c|}{Leptons}& \multicolumn{3}{c}{Quarks} \\
  \cline{2-7}
  Generation & \multicolumn{1}{|c}{Particle} & Mass & \multicolumn{1}{c|}{Electric Charge} & Particle & Mass & Electric Charge \\
  \hline\hline
  \multirow{2}{*}{1} & \Pem & 511 \keV & -1 & \Pqu & 2.3 \MeV & $+\frac{2}{3}$ \\
  & \Pgne & $\sim$0 & 0 & \Pqd & 4.8 \MeV & $-\frac{1}{3}$ \\
  \hline\hline
  \multirow{2}{*}{2} & \Pgmm & 105.7 \MeV & -1 & \Pqc & 1.275 \GeV & $+\frac{2}{3}$ \\
  & \Pgngm & $\sim$0 & 0 & \Pqs & 95 \MeV & $-\frac{1}{3}$ \\
  \hline\hline
  \multirow{3}{*}{2} & \Pgtm & 1.777 \GeV & -1 & \Pqt & 173.2 \GeV & $+\frac{2}{3}$ \\
  & \Pgngt & $\sim$0 & 0 & \Pqb & 4.18 \GeV & $-\frac{1}{3}$ \\
  \hline\hline
  \end{tabular}
\end{table}

\begin{table}
  \caption[The fundamental spin-1 vector bosons observed in nature and the force which they mediate.]{The fundamental spin-1 vector 
bosons observed in nature and the force which they mediate~\cite{pdg}.}
  \label{tab:bosons}
  \begin{tabular}{lccc}
    \hline
    \hline
    Force & Particle & Mass & Electric Charge \\
    \hline\hline
    Electromagnetism & \Pgg & 0 & 0 \\
    \hline
    \multirow{2}{*}{Weak} & \PWpm & 80.4 \GeV & $\pm 1$ \\
    \cline{2-4}
    & \PZ & 91.2 \GeV & 0 \\
    \hline
    Strong & g & 0 & 0 \\
    \hline
    \hline
  \end{tabular}
\end{table}

\subsection{Gauge symmetries}

The derivation in this section follows that in~\cite{ewk-int}. Consider a Lagrangian containing a fermionic field, $\psi$,

\begin{equation}
\mathcal{L} = \psi(x)^{\dagger} (i\gamma^{\mu}D_{\mu} - m)\psi(x) = \psi(x)^{\dagger} (i\cancel{D} - m)\psi(x)
\end{equation}

where $\psi(x)^{\dagger}$ is the hermitian conjugate of $\psi(x)$, $D_{\mu}$ is the \emph{covariant derivative} and
$\gamma^{\mu}$ are the Dirac matrices. Consider a local gauge symmetry defined by symmetry operator $U(x)$ such that

\begin{align}
\psi(x) &\rightarrow U(x)\psi(x), \\
\psi^{\dagger}(x) &\rightarrow \psi^{\dagger} (x) U^{\dagger} (x).
\end{align}

In the remainder of this section the dependence on x is implicit. In order to ensure gauge invariance the 
covariant derivative must transform as 

\begin{equation}
\label{equ:cov_deriv}
D_{\mu}\psi \rightarrow U D_{\mu}U^{\dagger} U \psi \rightarrow U \psi.
\end{equation}

This may be achieved by introducing a vector \emph{gauge} field, $A_\mu$, such that 
$D_{\mu} = \partial_{\mu} - igA_{\mu}$. To satisfy Equation~\ref{equ:cov_deriv},
$A_{\mu}$ transforms under the \emph{adjoint action}

\begin{equation}
A_{\mu} \rightarrow U A_{\mu} U^{\dagger} - \frac{1}{g}(\partial_{\mu}U)U^{\dagger}
\end{equation}

and the Lagrangian is therefore gauge invariant. Having introduced the gauge field may now be added to the Lagrangian,

\begin{equation}
\mathcal{L} = \psi^{\dagger} (i\cancel{D} - m)\psi - \frac{1}{4}F^{\mu,\nu}F_{\mu\nu},
\end{equation}

where $F^{\mu\nu}$ is the field strength tensor of the vector field,

\begin{equation}
F_{\mu\nu} = - \frac{1}{g}\left[D_{\mu},D_{\nu}\right] = \partial_{\mu}A_{\nu} - \partial_{\nu} A_{\mu} - g\left[A_{\mu},A_{\nu}\right].
\end{equation}

By writing the fields and covariant derivatives in terms of the generators of the group, $t_{a}$, such
that $A_{\mu} = A^{a}_{\mu} t_{a}$, the field strength tensor can be written as 
\begin{equation}
F^{a}_{\mu\nu}t_{a} = \left(\partial_{\mu}A^{a}_{\nu} - \partial_{\nu} A^{a}_{\mu} - g f_{bc}^{a}A^b_{\mu}A^c_{\nu}\right)t_a
\end{equation}
where $f_{bc}^{a}$ are the structure constants of the group defined by the 
commutation of the group generators. 

The gauge symmetries of the Lagrangian therefore result in massless vector 
bosons that are \emph{mediators} of the resultant forces. The properties and 
the forces, such as whether the mediators are self interacting, depend on the 
details of the gauge symmetry. In the next section the gauge symmetries
of the SM are discussed. 

% These may be non zero if the group is non-abelian and will introduce
% cubic and quadratic self interactions among the gauge fields.
%
% \begin{equation}
% \left[t_b,t_c\right] = f_{bc}^{a}T_{a}. 
% \end{equation}

% The Lagrangian may then by expanded as
% \begin{equation}
% \mathcal{L} = -\frac{1}{4}F_{\mu\nu}F^{\mu\nu} + i\psi^{dagger}\gamma^\mu D_{\mu}\psi
% \end{equation}

\subsection{Gauge symmetries of the standard model}
\label{sec:sm-gs}
The gauge symmetry group of the standard model is given by~\cite{ewk-int}
\begin{equation}
G_{SM} = SU(3)_{c}\otimes SU(2)_{L}\otimes U(1)_{Y}.
\end{equation}

The $SU(3)_c$ gauge symmetry is unbroken and therefore the associated
\emph{strong force} is mediated by a massless vector boson, the gluon.
$SU(3)_c$ is generated by the eight Gell-Mann matrices and there are therefore eight colour charges. 
The quarks carry a single colour charge, the gluon is double charged and all fundamental 
other particles in the SM are colour singlets. The gluon is self interacting which makes
the strong force short range. Due to \emph{charge screening} the strong coupling constant, $\alpha_s$, 
reduces with energy. This \emph{asymptotic freedom} requires quarks to be confined into colour singlets
of two (mesons) or three (baryons) quarks into hadrons. The energy required to separate these colour singlets 
is sufficient to generate additional quark/antiquark pairs (\emph{hadronisation}). High
energy collisions of protons, such as those at the LHC, result in the production of highly 
collimated emissions of hadrons (\emph{jets}) due to the liberation of quarks~\cite{salam}.

The gauge symmetry of the electroweak sector of the SM is $SU(2)_L\otimes U(1)_Y$. These symmetries lead
to the electromagnetic and weak forces mediated by the W/Z and $\gamma$ bosons respectively.

The Lagrangian for this sector may be written as 
\begin{equation}
\mathcal{L}_{Ewk} = \mathcal{L}_{gauge} + \mathcal{L}_{fermion} + \mathcal{L}_{Higgs} + \mathcal{L}_{Yuk}
\end{equation}

where the terms will be discussed in this section. The weak interaction distinguishes
between left and right handed chiralities and therefore the fermionic fields are split into
$\psi_{L/R} = (1\pm\gamma^5)\psi$. The fermion term may then be written for each of the 
three generations of quarks and leptons as,

\begin{equation}
\mathcal{L}_{fermion} = i \psiL^{\dagger}\cancel{D}\psiL + i \psiR^{\dagger}\cancel{D}\psiR.
\end{equation}

The left handed fields transform as doublets under the $SU(2)_L$ symmetry
while the right handed field are singlets.
For the first generations of quarks and leptons $\psi$ may therefore be written as

\begin{align}
\qquad\qquad\qquad\psiL =\,&{\begin{pmatrix} u_L \\ d_L \end{pmatrix}},\, &{\begin{pmatrix} \nu_{e\,L} \\ e_L \end{pmatrix}},\nonumber\qquad\qquad\qquad\\
\qquad\qquad\qquad\psiR = \,&\,u_R,d_R,\, &\,e_R.\qquad\qquad\qquad
\end{align}

The generators of $SU(2)_L$ are $T^i = \tau^i/2$, where $\tau^i$ are the three Pauli matrices. 
The covariant derivative therefore acts on the left and right handed components of $\psi$ as

\begin{equation}
D_{\mu} \psiL = (\partial_\mu + igW^i_{\mu}T^i + ig'YB_\mu)\psiL \, D_{\mu} \psiR = (\partial_\mu + ig'YB_\mu)\psiR
\end{equation}

where g and g' are the coupling constants of the $SU(2)_L$ and $U(1)_Y$ groups respectively, 
$W^i$ are the three gauge bosons that couple to the weak-isospin, T, and B is the gauge boson
coupling to hypercharge, Y. The hypercharge values are chosen such that the sum of the hypercharge and 
the third component of weak-isospin corresponds to the electric charge.
\begin{equation}
\label{equ:charge}
Q = T^{3} + Y
\end{equation}

For the gauge section of the lagrangian, the $SU(2)_L$ and U(1) symmetries give rise to two field strength tensors
\begin{align}
G^i_\mu\nu &= \partial_\mu W^i_\nu - \partial_\nu W^i_\mu - g \epsilon^{ijk}W^j_\mu W^k_\nu,\\
F_\mu\nu &= \partial_\mu B_\nu - \partial_\nu B_\mu.
\end{align}

Leading to

\begin{equation}
\mathcal{L}_{gauge} = -\frac{1}{4}F^{\mu\nu}F_{\mu\nu} - \frac{1}{4}G^{i\mu\nu}G^{i}_{\mu\nu}
\end{equation}

The Lagrangian containing these symmetries, however, does not correspond to the observed universe. No mass terms
are included for either the gauge bosons or fermions. These cannot be included directly as any such term breaks 
the gauge invariance and must therefore must be introduced via symmetry breaking.

The Higgs sector of the Lagrangian is given by,

\begin{equation}
\label{equ:higgs-lagrangian}
\mathcal{L}_{Higgs} = (D^{mu}\phi)^{dagger}(D_{\mu}\phi) - V(\phi)
\end{equation}

where the complex scalar field, $\phi$ is in the spinor representation of $SU(2)_L$, the covariant derivative of $\phi$ is

\begin{equation}
D_{\mu} \phi = (\partial_\mu + igW^i_{\mu}T^i + ig'\frac{1}{2}B_\mu)\phi 
\end{equation}

and the potential, V, is given by

\begin{equation}
V(\phi) =  - \mu^2\phi^{\dagger}\phi + \lambda \left(\phi^{\dagger}\phi\right)^2
\end{equation}

where $\mu^2$ and $\lambda$ are positive constants. The potential is minimised for any $\phi$ satisfying
$\phi^{\dagger}\phi = \frac{\mu^2}{2\lambda} \equiv v^2/2$. The minimum may be chosen as

\begin{equation}
<\phi> =  - \frac{1}{\sqrt{2}}\begin{pmatrix} 0 \\ v\end{pmatrix}.
\end{equation}

From Equation~\ref{equ:charge}, this scalar field has zero charge. The symmetry group has thus been broken from
$SU(2)_L\otimes U(1)_Y \rightarrow U(1)_Q$. The scalar field may be expanded around this minima as

\begin{equation}
\label{equ:phiExp}
\phi =  - \frac{1}{\sqrt{2}}\begin{pmatrix} 0 \\ v + h\end{pmatrix}.
\end{equation}

Identifying the physical W, Z and $\gamma$ (labelled A) gauge bosons as,

\begin{equation}
W^{\pm} = \frac{1}{\sqrt{2}} (W^1_\mu \mp i W^2_\mu), \begin{pmatrix} Z_\mu \\ A_\mu\end{pmatrix} = \begin{pmatrix} cos(\theta_W) & -sin(\theta_W) \\ sin(\theta_W) & cos(\theta_W)\end{pmatrix} \begin{pmatrix} W^3_\mu \\ B_\mu\end{pmatrix}
\end{equation}

where $\theta_W \equiv \text{atan}(g'/g)$ is the weak mixing angle. From Equation~\ref{equ:phiExp} and
the first term of Equation~\ref{equ:higgs-lagrangian} the masses may be identified as

\begin{equation}
M_W^\pm = \frac{1}{2}gv, M_Z = \frac{gv}{2cos\theta_W}, M_A = 0
\end{equation}

The masses of the fermions are derived from the Yukawa term in the Lagrangian, 
\begin{equation}
\mathcal{L}_{Yuk} = - \frac{1}{\sqrt(2)}(v + H)(f_mn e_L^{\dagger}e_R + h_{mn} d_{Lm}^{\dagger}d'_{Rn} + k_{mn} u_{Lm}^{\dagger}u_{Rn} + \text{hermitian conjugate}
\end{equation}

where $f_{mn}$, $h_{mn}$ and $k_{mn}$ are the Yukawa coupling matrices between the different generations. These may be diagonalised via unitary transformations
to generate mass terms for the quarks and leptons. The neutrino is massless in the SM as there is no right handed component.
This contradicts the observation of oscillations between neutrino flavours~\cite{neutOsc},
however, extensions which predict non-zero neutrino masses are possible~\cite{neutM}. 

The same unitary transformations also introduce mixing between quark generations from the 
terms including covariant derivatives in the Lagrangian. The mixings are summarised in the 
CKM matrix and occur as no basis of mass eigenstates is simultaneously diagonal for 
up and down type quarks~\cite{CKM}. Flavour changing
interactions between quarks are mediated by the $W^{\pm}$ boson while interactions via 
the Z and $\gamma$ bosons are flavour preserving. There are no flavour changing interactions
predicted for the leptons.

\section{Physics beyond the standard model}

The SM describes the properties and interactions between all known particles. These properties 
have been measured and the predictions of the SM verified in a multitude
of experiements. However, the theory cannot be complete 
as there remain fundamental problems which the standard model does not resolve.
Several of the largest such problems are described below.

On a theoretical level, a renormalisable theory of gravity cannot be included within 
the SM~\cite{gravRenorm}. While negligible at electroweak energy
scales, quantum gravitiational effects become increasingly important as
energies approach the planck scale, $M_{\text{planck}} \sim 10^{18} \GeV$. 

Measurements from cosmology and astrophysics have highlighted several areas
the SM fails to adequately describe Nature. The vacuum energy density 
of the universe (\emph{cosmological constant}), $\Lambda$, 
may be expected to be approximated by $M^4_{\text{planck}}$ by dimensional 
arguments. However, cosmological measurements of $\Lambda$ imply 
$\Lambda/M_{\text{planck}}^4 \sim 10^{-120}$~\cite{cosConst}. This discrepancy
is related to the failure to include gravity in a quantum field theory.

Astrophysical observations~\cite{WIMP} imply the existance 
of a weakly-interacting particle (WIMP) that forms the majority of the mat
ter within the universe. The SM has no viable candidates for 
this particle. In addition, the matter-antimatter asymmetry observed in the universe requires 
charge-parity violating processes far in excess of those occuring in the SM.

The discovery of the Higgs boson at $m_{h}\sim125\GeV$ raises the \emph{heirarchy problem}. In QFT, 
the corrections to $m_h$ are related to the heaviest particle in the theory and therefore
one may expect $m_h \sim M_{\text{planck}}$. In addition, quantum corrections from SM particles
require fine-tuning in the SM to stabilise $m_h$ (the \emph{technical heirarchy problem}, see Section~\ref{sec:natSUSY}).

These problems form the motivation for the existance of a Beyond standard model (BSM) physics model that
can resolve some or all of the oustanding issues in the SM. Supersymmetry (SUSY) is a particularly
well motivated BSM theory that can resolve the hierarchy problem, provide a DM candidate
and may include a quantum theory of gravity.

\section{Supersymmetry}

There are many possibilities for extending the SM including new particles/interactions, new (internal) gauge symmetries, 
extra spacial dimensions and/or new space-time (external) symmetries. SUSY is 
an external symmetry that relates fermions and bosons by extending the Poincar\'{e} algebra~\cite{SUSYC}. 
The Coleman-Mandula theorem shows that any such extension through new bosonic generators (such as those responsible for Lorentz 
transformations and translations) forbids non-zero scattering amplitudes. 
Supersymmetry therefore introduces fermionic generators, Q, such that (heuristically),

\begin{equation}
Q\ket{\text{Boson}} \sim \ket{\text{Fermion}} \quad Q\ket{\text{Fermion}} \sim \ket{\text{Boson}}.
\end{equation}

Particles connected by the SUSY generator are in \emph{supermultiplets}. This generator commutes with all internal symmetries 
of the SM and so particles within each supermultiplet will have identical electric, weak isospin and colour charges.
In addition, the SUSY generator also commutes with the mass operator and therefore 
particles within each supermultiplet have identical masses. The supersymmetric partners of the 
known SM particles \emph{sparticles} have not been observed and therefore SUSY must be a broken symmetry.

%
% Such sparticles are within reach of the 13\TeVket particle collisions at the 
% LHC (see Chapter~\ref{cha:detector}) 
% In order that these be in experimental reach of the LHC the SUSY 
% breaking scale cannot be greater than $\sim \TeV$ scale. There is no fundamental
% requirement for this scale, however, it is well motivated from the requirement 
% that SUSY provides a natural solution to
% the technical heirarchy problem (see Section~\ref{sec:natSUSY}).

\subsection{The minimally supersymmetric standard model}

The minimally supersymmetric standard model (MSSM) is the simplest SUSY extension of the standard model~\cite{SUSYP}.
The fermions in the standard model are \emph{chiral} (left and right handed pieces transform differently)
with two fermionic degrees of freedom per helicity state. The number of bosonic and fermionic degrees 
of freedom must be equivalent in each supermultiplet and therefore the simplest supermultiplet
is the \emph{chiral} supermultiplet containing the fermion and two real scalar fields.

The gauge bosons in the SM (before electroweak SUSY breaking) are massless spin-1 vector bosons
and therefore have two bosonic degrees of freedom (one per helicity state) and therefore
can be included in a \emph{gauge} supermultiplet with a spin-1/2 fermionic partner (\emph{gaugino}).

The SM Higgs does not simply reside in a chiral supermultiplet as this would introduce a 
gauge anomoly. This may be avoided by including two Higgs chiral supermultiplets with 
opposite hypercharge. The SM Higgs is a linear combination of the neutral components 
of these supermutiplets.

\begin{table}[!h]
  \centering
  \caption{Chiral supermultiplets in the MSSM~\cite{SUSYP}}
  \label{tab:chiral}
  \begin{tabular}
    {ccc}
    \hline\hline
    Name& spin-0 & spin-1/2 \\
    \hline
    \multirow{3}{*}{squarks, quarks (3 families) }& ($\tilde{u}_L\,\tilde{d}_L$) & ($u_L\,d_L$) \\
    & ($\tilde{u}^{*}_R$) & ($u_R^{\dagger}$) \\
    & ($\tilde{d}^{*}_R$) & ($d_R^{\dagger}$) \\
    \hline
    \multirow{2}{*}{sleptons, leptons (3 families) }& ($\tilde{\nu}\,\tilde{e}_L$) & ($\nu_L\,e_L$) \\
    & ($\tilde{e}^{*}_R$) & ($e_R^{\dagger}$) \\
    \hline
    \multirow{2}{*}{Higgs, higgsinos}& ($H_u^{+}\,H_u^{0}$) &  ($\tilde{H}_u^{+}\,\tilde{H}_u^{0}$) \\
    & ($H_d^{0}\,H_d^{-}$) &  ($\tilde{H}_d^{0}\,\tilde{H}_d^{-}$) \\
  \end{tabular}
\end{table}

\begin{table}[!h]
  \centering
  \caption{Gauge supermultiplets in the MSSM~\cite{SUSYP}}
  \label{tab:vector}
  \begin{tabular}
    {ccc}
    \hline\hline
    Name& spin-0 & spin-1/2 \\
    \hline
    gluino, gluon & $\tilde{g}$ & g \\
    winos, W bosons & $\tilde{W}^{\pm}\,\tilde{W}^0$ & $W^{\pm}\,W^{0}$ \\
    bino, B boson & $\tilde{B}^0$ & g \\
  \end{tabular}
\end{table}

The particle content of the MSSM (before symmetry breaking) 
is summarised in Table~\ref{tab:chiral} for the chiral supermultiplets
and Table~\ref{tab:vector} for the gauge supermultiplets. The neutral higgs and electroweak 
gaugino sectors combine to form mass eigenstates labelled \emph{neutralinos} 
(\chiz), while the charged higgs and electroweak gaugino sectors combine
to form \emph{charginos} (\chip). Gravity may also be included in the theory
by adding a spin-2 graviton and a spin-3/2 superparticle
called the gravitino~\cite{SUSYP}. 

The mass scale of the sparticles is dependant on the mechanism of SUSY breaking.
As discussed below, naturalness arguments motivate the lightest sparticle masses 
at the \TeV~scale.

\subsection{Natural supersymmetry}
\label{sec:natSUSY}
In the SM, the Higgs mechanism requires couplings to each fermion, f, of the form $-\lambda_{f}f^{\dagger}f\phi$. If BSM phyiscs
is expected to alter the high energy behaivour of the theory at a scale $\LUV$ the quantum corrections 
to the Higgs mass, $\delta m^2_{H}$ will be 

\begin{equation}
\Delta m^2_H =  \frac{|\lambda_f|^2}{8\pi^2}\left[-\Lambda^2_{\text{UV}} + 6 {m_f}^2\log\frac{\LUV}{m_f}\right] + \mathcal{O}\frac{1}{\LUV^2}
\end{equation}

which is quadratically divergent~\cite{HMSSM}. The correction is proportional to the size of the coupling of the Higgs to the fermion 
and is therefore largest for the top quark in the SM. Assuming the cut-off scale is at $M_{\text{planck}}$, the quantum 
corrections must be fine-tuned to $\sim 30$ orders of magnitude to predict the observed Higgs mass. 

In a BSM theory containing complex scalars, s, the Lagrangian will gain a term $-\lambda_{s}|\phi|^2|s|^2$.
Assuming for simplicity each component of the complex scalaar has mass $m_s$, the correction to the Higgs mass will be

\begin{equation}
\Delta m^2_H =  \frac{\lambda_s}{8\pi^2}\left[\Lambda^2_{\text{UV}} - 2 {m_s}^2\log\frac{\LUV}{m_s}\right] + \mathcal{O}\frac{1}{\LUV^2}
\end{equation}

If SUSY is a symmetry of nature, each fermion can be related to a complex scalar such that $\lambda_f^2 = -\lambda_s$ 
and therefore the correction to the Higgs mass will be 
%The overall correction to the Higgs boson mass will therefore be (neglecting terms not dependant on \lambda)

\begin{equation}
\label{equ:corrHiggsSusy}
\Delta m^2_H =  \frac{\lambda_s}{4\pi^2}\left[\Lambda^2_{\text{UV}} - 2 {m_s}^2\log\frac{\LUV}{m_s}\right],
\end{equation}

and the quadratic divergence is naturally cancelled. In unbroken SUSY, the logarithmic divergence also vanishes. 
Experimental observations imply the sparticle masses cannot be equal to their SM partners and
therefore an additional SUSY breaking term is required in the lagrangian. This must be a \emph{soft} breaking,
at a mass scale such that $\lambda_f^2 = -\lambda_s$ still holds to cancel the quadratic divergence.
To avoid fine funing the logarithmic term in~\ref{equ:corrHiggsSusy}, the lightest sparticles should have 
masses around the $\TeV$ scale. 

\subsection{R-Parity}

The general MSSM \emph{super potential} containts terms which violate baryon number, B, and lepton number, 
L. Such terms lead to predictions of photon decay, which is constrained experimentally to $> 10^{34}$ years~\cite{protonDecay}, 
on the order of seconds. To forbid such a process an extra symmetry called \emph{R parity} defined as 
\begin{equation}
R \equiv (-1)^{3(B-L)+2S} = 
\begin{cases}
+ 1\quad \text{SM particles}\\
- 1\quad \text{superpartners}
\end{cases},
\end{equation}
where S is the spin~\cite{SUSYP}. R parity has several physical implications including a stable lightest supersymmetric 
particle (LSP). This particle must be electrically and colour neutral from cosmological constraints.
Typically, it is taken to be the lightest neutralino in the MSSM and is an ideal WIMP candidate for DM. 
In addition R parity also requires superpartners 
to be formed in pairs in collisions and that each superpartner decays to another superpartner
in a chain that must with the LSP. The LSP does not interact with any detector and therefore
such collisions will produce signatures containing significant momentum imbalance.

\subsection{Experiemental SUSY signatures at the LHC}

The LHC provides proton collisions at $\sqrt{s} = 13 \TeV$. The sparticles have identical charges under 
the symmetries of the SM and, therefore, in a generic (natural) MSSM model 
coloured sparticles may be expected to have the highest production 
cross-section~\cite{susyprod}. Assuming R-parity, these sparticles can be produced through


\item quark-antiquark scattering or gluon fusion to produce a gluino or squark pair
\item quark and gluon scattering to produce a squark and gluino
\item quark-quark scattering to produce a squark pair

Each of the sparticles will decay in a chain to the LSP. In the decay, 
coloured SM particles are produced which will hadronise into jets.
The signature produced by such processes will therefore be significant hadronic activity in 
the form of jets as well as momentum imbalance from the LSP. The energy in
the final state and the momentum carried by the LSP are dependant on the masses
of the sparticles.

In natural SUSY, the lightest sparticle is expected to be the partner of the top
as the top makes the largest contribution to the Higgs divergence. Therefore a signature
of multiple top quarks is expected in the final state. As discussed in Section~\ref{topbg}, top quarks
decay primarily to a b quark and W boson. The sparticles in a natural MSSM theory may therefore be expected
to produce a final state containing multiple b quarks.

The SM contains processes which produces events with similar signatures to 
those of hadronic SUSY described above. These are described fully in Section~\ref{sec:ewk-background-intro},
and, in order to be sensitive to SUSY signals, such backgrounds must be mitigated 
and any residual backgrounds measured. The search described in this thesis relies on several
techniques to mitigate and measure backgrounds as described in Chapters~\ref{cha:alphat} and~\ref{cha:backgroundPrediction}
respectively.

\subsection{Simplified models}

The MSSM contains up to 105 free parameters which affect the production and decay modes of the sparticles.
It is therefore not possible to interepret the results of searches for SUSY in the full MSSM. 
In previous searches for SUSY, the constrained MSSM (the CMSSM), which leaves five free parameters, was 
used to interpret~\cite{SMS} the results. However, while parametrically simple, the decay chains of sparticles 
in the CMSSM are complicated, making such interpretations model dependant. 

To evaluate model independant reach in terms of gluino and squark masses, searches are interpreted using 
simplified models that are defined by a fixed set of production and decay modes.
Simplified models are effective models where the majority of the particle content in the theory 
are decoupled at high masses. Each of the simplified models considered in this thesis involves
the direct pair production of only one sparticle type which then decays directly to the LSP.

The T2 simplified models are simplified versions of squark-antisquark production. Each squark undergoes
a two-body decay to a quark and the LSP. In this thesis direct bottom squark production followed
by decay to bottom quark and LSP as well as direct top squark production followed by decay to top
quark and LSP are considered.

The T1 simplified models are a simplified versions of gluino pair production. Each gluino undergoes a three-body 
decay to a quark-antiquark pair and the LSP. In this thesis, decays to top-antitop as well as bottom-antibottom
pairs are considered. As the three body decay proceeds via the virtual squark these models 
are referred to as gluino-mediated squark production.

\begin{figure}[h!]
  \begin{center}
    \subfigure[Gluino mediated bottom squark production]{
      \includegraphics[width=0.3\textwidth]{Figures/theory/T1bbbb_feyn}
      \label{fig:T1bbbb_feyn}
    } ~~
    \subfigure[Glunio mediated top squark production]{
      \includegraphics[width=0.3\textwidth]{Figures/theory/T1tttt_feyn}
      \label{fig:T1tttt_feyn}
    } \\
    \subfigure[Direct bottom squark production]{
      \includegraphics[width=0.3\textwidth]{Figures/theory/T2bb_feyn}
      \label{fig:T2bb_feyn}
    } ~~
    \subfigure[Direct top squark production]{
      \includegraphics[width=0.3\textwidth]{Figures/theory/T2tt_feyn}
      \label{fig:T2tt_feyn}
    } 
    \caption{
      Graphical representation of the production and decay of supersymmetric particles 
      for the simplified models considered in this thesis~\cite{SMS}.
    }
    \label{fig:simplified-models-feyn}
  \end{center}
\end{figure}

Figure~\ref{fig:simplified-models-feyn} displays the decay chains for the models considered in this theis. 
In the simplified models the masses of the heavy sparticle and of the LSP are free parameters.
The topology of the model is crucial in determining the reach of experimental searches. 
In cases where the mass splitting is large the final state typically contains significant
momentum imbalance and hadronic energy. If the mass splitting is small, (\emph{compressed} models),
the momentum imbalanca and hadronic energy is suppressed.

The simplified models may be used to approximate the impact of searches on a complete theory 
by determining the relevant simplified topologies contained in the full theory. The impact
of searches for many different types of signatures may be combined using programes 
such as FastLim~\cite{Fastlim}.
% Simplified models with an intermediate particle in the decay may also be considered.
% For the work contained in this thesis




