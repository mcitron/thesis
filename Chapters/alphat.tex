\chapter{The \alphat search}

Hadronic searches for new physics aim to exploit strong force modes of 
production and decay. Such modes may be expected to dominate in the 
proton-proton collisions at the LHC. The \alphat~analysis relies on a
final state including jets and significant \met~to search for new physics.

The \alphat~analysis uses dedicated dimensionless variables to effectively reject the
dominant QCD multijet background. A robust data-driven estimate of the remaining
backgrounds and systematic uncertainties reduces the reliance on simulation.
During Run 1, this strategy has been used for several searches for supersymmetry at
both $\sqrt{s} = 7 \TeV$ and $\sqrt{s} = 8 \TeV$ as well as a range of luminosities
\cite{alphaT1,alphaT2,alphaT3,alphaT4}. As detailled in this section, substantial 
improvements to the sensitivity and acceptance of the search have been made for Run 2.

\section{Introduction to the \alphat analysis}

In order to achieve sensitivity to a wide range of new physics signatures
searches for new physics must reject backgrounds while maintaining signal acceptance.
For any hadronic search, QCD multijet processes will dominate. The
\alphat analysis uses selections based on key discriminating variables,
described in Section~\ref{sec:important-variables}, to reduce such backgrounds
to the percentage level. Additional hadronic and cleaning selections, summarised in
Sections ?? and ??, define the signal region. 

The determination of residual multijet backgrounds as well as backgrounds with 
genuine \met~relies on data driven techniques described in Sections ?? and ?? respectively.
These data driven predictions use signal depleted \emph{control regions} enriched in a particular
background process (or related process). The selections in
these control regions closely follow those in the signal region and are detailed in 
Section ?? and ??.

As discussed in Section~\ref{sec:triggerUpgrade}, an effective trigger strategy is 
critical to ensure acceptance to models which may, for example, have relatively low
\scalht~but significant~\mht. The trigger strategy for the signal and control regions
is discussed in Section ??.

To inclusively optimise sensitivity, the events that pass the signal region requirements 
must be categorised to allow significant signal contributions for a wide range of models. 
Several variables are used, as detailed in Section ??, to categorise the signal region. 
The characterisation of these variables using the control
regions is shown in Section ??.

\section{backgrounds}
\subsection{The QCD multijet background}
\label{sec:qcd-background-intro}
The QCD multijet background dominates for hadronic searches at the LHC. The cross section for such
inelastic scattering has been measured at $13\TeV$ as $\sim78.1mb$~\cite{inelast}. These events
are typically balanced dijet events, though higher jet multiplicities are also possible. 
Unlike the signal processes for which the $\alphat$ search is sensitive, these events have no
true \met. Fluctuations in detector response and reconstruction can cause a small fraction 
of the QCD events to gain significant \emph{fake} \met. As the total QCD cross section is 
around 6-7 orders of magnitude larger than that of the electroweak background these events with
fake \met become a dominant background. The main detector and reconstruction mechanisms that may introduce 
this \met are summarised below:
\begin{itemize}
\item Detector inefficiencies due to regions with reduced or no response (\emph{dead cells}) can cause 
a significant proportion or all of the energy of any incident physics object to be lost. If the true event 
is balanced, when losses occur due to detector inefficient the \met~vector will point approximately in the
$\phi$ direction of the problematic region.
\item Misreconstruction due to effects such as tracking errors, incorrect object identification and
under/over correction of jets when calibrating. This may apply to one or more objects in the event. If the 
underlying event is balanced the \met~vector will typically point in or opposite to the direction 
of the misreconstructed object in cases of under or over estimation of the energy respectively.
\item Additional energy can be added in the event due to effects including \emph{hot cells} which consistently records energy 
regardless of incident particles in the ECAL or HCAL, spontaneous discharges and direct particle interactions with detector
electronics or photomultipliers
\item The \emph{beam halo} of charged particles around the LHC beam, caused mainly by proton scattering off LHC collimators, 
may interact with muon chambers causing fake muons to be reconstructed or deposit energy in the calorimeters as they traverse 
the detector~\cite{beam_halo}. The beam halo has the largest effect for $\phi = 0$ and $\phi=\pi$ as the constituent particles
tend to lie within the plane of the LHC ring.
\item Imbalance nay be introduced by acceptance effects. If physics objects are excluded from the calculation
of energy sums due to thresholds in quantities such as $\pt$ or $\eta$ \met will be introduced. These thresholds are typically 
required due to imperfect detector coverage or to remove objects reconstructed due to effects such as detector 
noise or pileup.
\end{itemize}
In addition to QCD multijet events with fake \met introduced by such detector and reconstruction effects,
QCD scattering processes may produce events containing true \met. This is due to the rare production of heavy flavour 
quarks which decay via leptons and neutrinos. Such events pass hadronic selection as the leptons
are typically confined within a jet cone and so fails isolation requirements. In such cases the \met~vector
is typically closely aligned with the $\phi$ direction of the jet.
\subsection{ewk}
\label{sec:ewk-background-intro}
%should this be in theory section?

\section{Suppression of the QCD multijet background}
\label{sec:important-variables}
\subsection{alphaT}

\subsection{bdphi}

\subsection{\mht/\met}
The \mht/\met cleaning cut is designed to reduce the contamination from balanced events
due to acceptance effects from kinematic and pseudorapidity thresholds, 
severe jet mismeasurement and particles not clustered into jets. The PF \met includes
all PF candidates and so the contamination from such events may be mitigated 
using a maximal threshold on the ratio of the \mht to the \met~.

\subsection{Forward jet veto}

The $\mht$ variable is made using jets with $\etaabs < 2.4$. Jets in the forward pseudorapidity 
region may introduce large $\mht$. A veto on any jet in the forward region with $p_T > 40\GeV$
is used to veto such events. 
\label{sec:fwd_jet_veto}
\section{Physics objects}
\subsection{Jets}

The jet collection is defined by PF jets (clustered from PF candidates) as defined in Section~\ref{sec:jet_reco} with
$\Delta R = 0.4$. The jets are corrected for pileup contributions using CHS and jet area corrections. Additional cleaning cuts are then applied
on the jet constituents as summarised in Table~\ref{tab:loose-jet-id}. These requirements are necessary to reject fake jets. The charged hadron
fraction cut of $> 0.1$ additionally rejects jets reconstructed from beam halo interactions.

A threshold of jet $\pt > 40\GeV$ is required for jets used in the analysis. As supersymmetric models tend to produce
relatively hard jets this provides a good efficiency while effectively rejecting QCD backgrounds. Jets in the signal 
region must satisfy the pseudorapidity requirement $\etaabs < 2.4$. The presence of any forward jets with $\etaabs > 2.4$
is used as a cleaning veto as described in Section~\ref{sec:fwd_jet_veto}. Additionally, the leading jet in the 
event must satisfy $\pt > 100\GeV$. 
\begin{table}[ht!]
  \caption{Jet identification requirements. \label{tab:loose-jet-id}}
  \centering
  \begin{tabular}{ ccc }
    \hline
    \hline
    Variable & cut & notes \\ \hline
    \multicolumn{3}{c}{$-3.0 < \eta_{\mathrm{jet}} < 3.0$} \\ \hline    
    Neutral Hadron Fraction & $<0.99$ & - \\
    Neutral EM Fraction & $<0.99$ & - \\
    Number of constituents & $>1$ & - \\
    Charged Hadron Fraction & $>0$ & only for $|\eta_{\mathrm{jet}}| < 2.4$ \\
    Charged Multiplicity & $>0$ & only for $|\eta_{\mathrm{jet}}| < 2.4$ \\
    Charged EM Fraction & $<0.99$ & only for $|\eta_{\mathrm{jet}}| < 2.4$ \\ \hline
    \multicolumn{3}{c}{$|\eta_{\mathrm{jet}}| > 3.0$} \\ \hline        
    Neutral EM Fraction & $<0.90$ & - \\
    Number of Neutral Particles & $>10$ & - \\
    \hline
    \hline
  \end{tabular}
\end{table}

\subsubsection{B-tagged jets}

Jets originating from bottom quarks are identified using the CSVv2 algorithm defined in Section~\ref{sec:btag}. 
The tagging efficiency for the working point of the discriminator of 0.89 is $\sim 67\%$ for a mistag rate
of $\sim 1\%$ for light quarks ($u$, $d$ and $s$ quarks) and gluons and a mistag rate of $\sim 10\%$ for
charm quarks.

\subsection{Photons}

Photons must be identified to define the \gj~control region and veto events in the signal and
muon control regions. Their reconstruction is described in Sections~\ref{sec:ele_pho_reco}~and~\ref{sec:particle_flow}. The photon isolation is 
ensured using PF relative isolation with a cone size of $\Delta R = 0.3$. Pileup contributions 
and mitigated using EA correction. Additional selections are summarised 
in Table~\ref{tab:photon-id-gamma} and an efficiency for photon identification of $\sim71\%$
is achieved. The number of fake photons passing selection is measured as $2-10\%$ depending on $\scalht$.

The kinematic selection for photons used to veto events in the signal region is $\pt > 25\GeV$
and $\etaabs < 2.5$. The photons used to define the control region must satisfy the tighter 
requirements $\pt > 25\GeV$ and $\etaabs < 1.4$ to ensure efficient trigger selection and
that the photon is contained within the barrel where it can be better reconstructed.

\begin{table}[ht!]
  \caption{Photon identification requirements.\label{tab:photon-id-gamma}}
  \centering
  \footnotesize
  \begin{tabular}{ ccc }
    \hline
    \hline
    Categories & \multicolumn{2}{c}{Barrel}   \\
    Working point  & Tight & Loose \\
    \hline
    Conversion safe electron veto & Yes & Yes  \\
    Single Tower H/E              & 0.05 & 0.05  \\
    $\sigma_{i\eta i\eta}$        & 0.0100 & 0.0102 \\
    PF charged hadron isolation   & 0.76 & 3.32  \\
    PF neutral hadron isolation   & 0.97 + 0.014 $\times$ $p_{\mathrm{T},\gamma}$ + 0.000019 $\times$ $p_{\mathrm{T},\gamma}^{2}$ & 1.92 + 0.014 $\times$ $p_{\mathrm{T},\gamma}$ + 0.000019 $\times$ $p_{\mathrm{T},\gamma}^{2}$  \\
    PF photon isolation           & 0.08 + 0.0053 $\times$ $p_{\mathrm{T},\gamma}$ & 0.81 + 0.0053 $\times$ $p_{\mathrm{T},\gamma}$ \\
    \hline
    \hline
    Categories & \multicolumn{2}{c}{Endcap}   \\
    Working point  & Tight & Loose \\
    \hline
    Conversion safe electron veto & Yes & Yes  \\
    Single Tower H/E              & 0.05 & 0.05  \\
    $\sigma_{i\eta i\eta}$        & 0.0268 & 0.0274 \\
    PF charged hadron isolation   & 0.56 & 1.97  \\
    PF neutral hadron isolation   & 2.09 + 0.014 $\times$ $p_{\mathrm{T},\gamma}$ + 0.000025 $\times$ $p_{\mathrm{T},\gamma}^{2}$ & 11.86 + 0.014 $\times$ $p_{\mathrm{T},\gamma}$ + 0.000025 $\times$ $p_{\mathrm{T},\gamma}^{2}$ \\
    PF photon isolation           &  0.16 + 0.0034 $\times$ $p_{\mathrm{T},\gamma}$ & 0.83 + 0.0034 $\times$ $p_{\mathrm{T},\gamma}$ \\
    \hline
    \hline
  \end{tabular}
\end{table}

\subsection{Electrons}

Electrons are identified to veto events in the signal and control regions. The full reconstruction is described in
Sections~\ref{sec:ele_pho_reco}~and~\ref{sec:particle_flow}. The isolation uses PF mini-isolation 
with a variable cone size of maximum radius $\Delta R = 0.2$. The requirements are summarised in 
Table~/ref{tab:ele-id}. An overall efficiency for electron selection of $\sim90\%$ is achieved.

The kinematic requirements for electrons used for vetoing are $\pt > 10\GeV$ and $\etaabs < 2.5$.
\begin{table}[h!]
  \caption{Electron identification requirements.\label{tab:ele-id}}
  \centering
  \footnotesize
  \begin{tabular}{ lcc }
    \hline
    \hline
    Categories                                               & Barrel    & EndCap    \\
    \hline
    $\Delta \eta_{In}$                                       & 0.0105   & 0.00814  \\
    $\Delta \phi_{In}$                                       & 0.115    & 0.182  \\
    $\sigma_{i\eta i\eta}$                                   & 0.0103    & 0.0301  \\
    H/E                                                      & 0.104    & 0.0897   \\
    d0 (vtx)                                                 & 0.0261    & 0.118  \\
    dZ (vtx)                                                 & 0.041    & 0.822  \\
    $\lvert(1/E_{\textrm{ECAL}} - 1/p_{\textrm{trk}})\rvert$ & 0.102     & 0.126  \\
    Missing hits (inner tracker)                             & 2         & 1         \\
    Conversion veto                                          & yes       & yes   \\
    \hline
    \hline
  \end{tabular}
  \end{table}
\subsection{Muons}

Muons are selected to define the single mu, $\mj$, and double mu, $\mmj$ control regions as well as 
for veto in defining the signal regions. The reconstruction is described in Section~\ref{sec:muon_reco}.
The isolation is defined using a PF relative isolation requirement of $I_{\text{PF}}^{\text{rel}} < 0.15$ 
with a cone size of $\Delta R = 0.4$ for muons in the control regions and using a PF mini-isolation requirement of
$I_{\text{PF}}^{\text{mini}} < 0.2$ with a maximum cone size of $\Delta R = 0.4$. The muons are pileup corrected
using an EA correction. An efficiency for muon selection of $\sim98\%$ is achieved.

The kinematic requirements for muons used for veto are $\pt > 10\GeV$ and $\etaabs < 2.4$, while muons 
selected in the control regions must satisfy $\pt > 30\GeV$ and $\etaabs < 2.1$. This ensures efficiency
for passing trigger requirements and that the muon is well reconstructed.


\subsection{Isolated tracks}

Isolated tracks are used to identify W boson decays and single prong decays of the $tau$ (see Section ??). 
They are selected from the charged PF candidates satisfying $\pt > 10\GeV$, $\Delta z(\text{track},\text{PV})$ and 
$I_{\text{PF}}^{\text{rel}} < 0.1$ with a cone of $\Delta R = 0.3$.

\subsection{Energy sums}

The $\met$ is computed from the magnitude of the vector sum of the transverse momentum of all PF candidates in
the event. This is type-1 corrected, as described in Section~\ref{sec:energy_sums_reco}, 
using PF jets with $\pt > 15\GeV$. In the control regions the object(s) used to define
the control region is not included the $\met$ calculation. The \mht~and \scalht~are defined
using the vector and scalar sum respectively of all PF jets satisfying $\pt > 40\GeV$ and $\etaabs < 2.4$.

\subsection{summary}

\section{Preselection}
\subsection{event filters}
\subsection{event vetoes}
\subsection{kinematic selections}
\subsection{summary}

\section{Trigger strategy}
\subsection{Hadronic signal region}
\subsection{Control regions}
%Should this be earlier?

\section{Analysis selections}
\subsection{Hadronic signal region}
\subsection{Hadronic control region}
\subsection{Photon control region}
\subsection{Muon control regions}
\subsection{summary}

\section{Categorisation}
\subsection{njet nb ht}
\subsection{mht}

%Move to later chapter?
\section{Characterisation of control regions}
\subsection{Yields and distributions for muon}
\subsection{Yields and distributions for dimuon}
\subsection{Yields and distributions for photon}
