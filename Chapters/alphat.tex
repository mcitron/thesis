\chapter{The \alphat search}
\label{cha:alphat}

The \alphat~analysis is a hadronic search for BSM physics 
targeting the pair-production of coloured SUSY particles that
decay to the weakly interacting LSP.

Exploiting a hadronic final state requires effective 
background suppression and residual background prediction. During Run 1, 
the \alphat~analysis strategy has been used for several searches for supersymmetry at
both $\sqrt{s} = 7\,\TeV$ and $\sqrt{s} = 8\,\TeV$, as well as a range of luminosities
\cite{alphaT1,alphaT2,alphaT3,alphaT4}. For Run 2, the selections and categorisation 
have been updated to improve the sensitivity and acceptance of the search; they are detailed
in this section.

The \alphat~analysis `signal region' is defined by selecting a final state 
including jets and significant \met~with no reconstructed leptons or photons. 
These selections provide sensitivity to hadronic SUSY models with significant \met
in the final state.

Background control is crucial in a search for BSM physics. To mitigate the otherwise 
dominant QCD multijet background, detailed in Section~\ref{sec:qcd-background-intro}, dedicated dimensionless variables, 
defined in Section~\ref{sec:important-variables}, are used to strongly suppress this process.
In defining the signal region, additional hadronic, kinematic and cleaning selections 
are used for further QCD and electroweak background suppression; these are described in 
Section~\ref{sec:had_sr}.

The determination of residual QCD multijet backgrounds as well as backgrounds with 
genuine \met~(described in Section~\ref{sec:ewk-background-intro}) relies on data-driven techniques 
described in Chapter~\ref{cha:backgroundPrediction}. These data-driven predictions use signal 
depleted `control regions' enriched in a particular background process 
(or related process). All events used in the \alphat~analysis pass
a common preselection described in Section~\ref{sec:presel}. Further selections in
these control regions closely follow those in the signal region; they are detailed in 
Section~\ref{sec:cr-sel}.

As discussed in Chapter~\ref{cha:triggerUpgrade}, an effective trigger strategy is 
critical to ensure acceptance for signals such as compressed SUSY models which can have 
relatively low \scalht~but significant~\mht. The trigger strategy for the signal and control regions
is discussed in Section~\ref{sec:ana-trigger}.

To inclusively optimise sensitivity, the events that pass the signal region requirements 
must be categorised to allow significant signal contributions for a wide range of models. 
Several variables are used, as detailed in Section~\ref{sec:cat}, to categorise the signal region. 
% The characterisation of these variables using the control
% regions is shown in Section~\ref{sec:char}.

\section{Backgrounds for hadronic searches}
\subsection{The QCD multijet background}
\label{sec:qcd-background-intro}
The QCD multijet background from inelastic scattering is dominant in hadronic searches at the LHC. 
The final state is typically a balanced dijet event with no weakly interacting particles 
in the final state. Higher jet multiplicity events are rarer but also possible.

While the events contain no `true'~\met from weakly interacting particles (except 
for `heavy flavour' QCD, see below),
fluctuations in detector response and reconstruction can cause a small fraction 
of the QCD events to gain significant `fake'~\met. These form a dominant background as 
the total QCD cross section is up to $\sim7$ orders of magnitude larger 
than that of the SM backgrounds containing true~\met (see Section~\ref{sec:ewk-background-intro})~\cite{inelast}.
The main detector and reconstruction mechanisms that introduce fake~\met are summarised below:

\begin{itemize}
\item Detector inefficiencies due to regions with reduced or no response (`dead cells') can cause 
a significant proportion of the energy of any incident physics object to be lost. If the true event 
is balanced, losses which are caused by detector inefficiency will contain a \met~vector which points 
approximately in the $\phi$ direction of the problematic region.
\item Misreconstruction due to effects such as tracking errors, incorrect object identification and
under/overcorrection of jets when calibrating. This may apply to one or more objects in the event. 
If the underlying event is balanced, the \met~vector will typically point in 
the direction of the misreconstructed object in cases of underestimation 
or in the opposite direction in cases of overestimation.
\item Additional energy can be added to the event due to effects such as `hot cells' that 
consistently record energy depositions without incident particles in the ECAL or HCAL, 
spontaneous discharges and direct particle interactions with detector
electronics or photomultipliers.
\item The `beam halo' of charged particles around the LHC beam, caused mainly 
by proton scattering off the LHC collimators, may interact with muon chambers, causing fake muons to be 
reconstructed, or may deposit energy in the calorimeters as they traverse the detector~\cite{beam_halo}. The beam halo has the largest 
effect for $\phi = 0$ and $\phi=\pi$ as the constituent particles tend to lie within the plane of the LHC ring.
\item Imbalance may be introduced by acceptance effects. If physics objects are excluded from the calculation
of energy sums due to thresholds in quantities such as $\pt$~or $\eta$, ~\met~will be introduced. These thresholds are typically 
required because of imperfect detector coverage or to remove objects reconstructed from effects such as detector 
noise or pileup.
\end{itemize}
In addition to QCD multijet events with fake~\met discussed above, QCD scattering processes 
may produce events containing true~\met. This is due to the rare production of heavy-flavour 
quarks which decay via leptons and neutrinos. These events pass hadronic selection as the leptons from 
heavy-flavour decay are usually confined within the jet cone and fail isolation requirements. 
In such cases the \met~vector is typically closely aligned with the $\phi$ direction of the jet.

\subsection{Electroweak backgrounds}
\label{sec:ewk-background-intro}
%should this be in theory section?
There are several SM processes which include \met~from neutrinos in the final state
and can form backgrounds for hadronic searches (labelled as electroweak backgrounds). The dominant electroweak
backgrounds are \wj, \ttj~and \zj. The $\alphat$ signal region, defined fully in 
Section~\ref{sec:had_sr}, selects purely hadronic events
containing reconstructed jets and~\met. An overview of the electroweak background processes 
and the mechanisms by which they may enter into the signal region is given below. 

\subsubsection{\zj background}

The largest background process in the signal region comes from Z boson production
in association with jets (\zj) followed by decay to neutrinos, \znunu. This process is 
an `irreducible' background as the neutrinos produce significant \met~in the final 
state with no associated leptons. Across the signal region, \znunu~contributes $\sim 50\%$ of 
the overall background. The decay of the Z boson to leptons, \zll~may also introduce $\met$ if 
one or both of the leptons is not reconstructed (lost). Due to the low probability of
losing both leptons in the event, this is a subdominant background, comprising $\sim 0.5\%$ of the signal region.

\subsubsection{\wj background}

The production of a W boson in association with jets (\wj) where the W boson decays leptonically
(\wl) comprises $\sim 40\%$ of the overall background in the signal region. 
Such events pass hadronic selection if the lepton is not reconstructed.
This lost lepton background may be introduced if the lepton falls out of acceptance,
fails isolation requirements or is otherwise misreconstructed. Both the lost lepton 
and the neutrino in the event may contribute to the~\met. 
% Typically, however, 
% the neutrino dominates the \met~as low $\pt$ leptons are more likely to be misreconstructed.
The branching fraction for a W boson decaying to a hadronically decaying $\tau$ is $\sim6\%$.
In such events the associated $\tau$ neutrino may introduce significant~\met. If the hadronic
decay products are reconstructed as a jet, these events may pass the signal region selection.
To reduce this background, a single isolated track veto is used to reject single-prong decays of the 
$\tau$~($\tau^{\pm}\rightarrow\pi^{\pm}+n\pi^{0}\nu$) which comprise $\sim70\%$ of the hadronic $\tau$ decays.

\subsubsection{\ttj and single top background}
The \ttj~and single top background comprise around $5\%$ and 0.5\% of the overall
background, respectively. On production, the top quark will predominantly decay via the 
weak force as \twb. The W boson may then decay hadronically, producing a final state
with multiple jets, or leptonically, producing a final state including a lepton and a neutrino. 

The final state in an event with \ttj production will include two bottom quarks as well as multiple 
jets and/or significant \met, depending on the decay. If both W bosons decay hadronically ($\sim45.7\%$ of \ttbar~decays),
up to six jets may be produced from the hadronisation of the quarks. In this case the $\met$
may be introduced via jet mismeasurement or from bottom quark decay via neutrinos. If one of the 
W bosons decays leptonically ($\sim45.8\%$ of \ttbar~decays), up to four jets may be expected
as well as~\met from the neutrino. Such events may enter hadronic selection if the lepton is lost or is
a hadronically deacying $\tau$.
Finally, if both W bosons decay leptonically ($\sim10.5\%$ of \ttbar~decays), the final state contains two 
leptons and significant \met. This mode is subdominant due to the low probability of losing both leptons
and the lack of jets in the final state. For all decay modes, the number of jets reconstructed in the event
may be increased by processes such as ISR/FSR and pileup.

The \alphat analysis categorises events using the number of reconstructed 
jets identified as originating from bottom quarks. The \ttbar background is particularly
enriched for $\nb \ge 2$, comprising $\sim60\%$ of the background for such events.

Single top production occurs mainly via t-channel in association with a quark, production in 
association with a W boson (tW), or s-channel in association with a bottom quark, with approximate
proportions 72.5, 24 and 3.5\% respectively. As for \ttj, the single top background is enhanced for 
categorisations with at least two reconstructed bottom quarks, rising to $\sim5\%$ of the total for
such events.

\subsubsection{Residual electroweak backgrounds}

In addition to the backgrounds discussed above, there are several processes which 
make minor contributions to the signal region (which can be enhanced
for particular categorisations). Such residual backgrounds include \ttbar~production in association with a vector
boson, \ttV ($\sim0.1\%$ of the total background and $\sim4\%$ of the total background for an $\nb \ge 2$ selection); 
pair production of vector bosons, ZZ, WZ and WW ($\sim1.4\%$ of the total background); 
production of a (lost) photon in association with jets ($\sim0.7\%$ of the overall background) and
leptonic decays of the Z with two lost leptons ($\sim0.5\%$ of the overall background).

\section{Suppression of the QCD multijet background}
\label{sec:important-variables}

Predicting the QCD backgrounds presents significant challenges, discussed in Section~\ref{sec:qcd-pred}, which
can introduce large uncertainty in the background estimation. A distinguishing feature
of the \alphat~analysis is the aim to mitigate this uncertainty by reducing the 
QCD multijet background to under 10\% of the total in each category of the analysis. 
This is achieved using selections on the dedicated variables \alphat, \bdphi, \mhtmet~and the forward jet veto. This section
describes how these variables reject QCD events by exploiting the topologies and features of such events. Further
event filters that specifically target events containing $\met$ introduced by
known detector problems and beam halo effects are discussed in Section~\ref{sec:event_filters}.

\subsection{\alphat}
The \alphat variable is designed to reject balanced events which gain significant~\met through
jet mismeasurement. The $\alpha$ variable was initially proposed in~\cite{Randall} and
converted into the transverse variable, $\alphat$, to allow use with hadronic collisions. The absolute
value of the $\met$ is sensitive to the detector and reconstruction effects discussed in Section~\ref{sec:qcd-background-intro}. 
The \alphat variable is designed to be dimensionless such that the topology of the event is used to reject 
QCD processes, regardless of the total value of the~\met.

For a dijet event, \alphat is defined as

\begin{equation}
\label{eq:alphat}
\alphat\, =\, \frac{\Et^{{\rm j}_2}}{M_\text{T}} \, ,
\end{equation}

\noindent where $\Et^{\rm j_2}$ is the transverse energy of the 
less energetic jet and $M_\text{T}$ is the transverse
mass of the dijet system, 

\begin{equation}
  \label{eq:mt}
  M_\text{T}\, = \,\sqrt{ \left( \sum_{i=1}^2 \Et^{{\rm j}_i}
    \right)^2 - \left( \sum_{i=1}^2 p_x^{{\rm j}_i} \right)^2 - \left(
      \sum_{i=1}^2 p_y^{{\rm j}_i} \right)^2} \, .
\end{equation}

\noindent where $\Et^{{\rm j}_i}$ is the transverse energy of jet ${\rm j}_i$ 
($\Et^{{\rm j}_i} = E^{{\rm j}_i}\sin\theta^{{\rm j}_i}$), and
$p_x^{{\rm j}_i}$ and $p_y^{{\rm j}_i}$ are the $x$ and $y$ components
of the transverse momentum of the jet. 

For events with three or more jets, a pseudo-dijet system is defined 
where all possible vectorial sums of the jets into two
pseudo-jets are considered. The combination into pseudo-jets 
with the smallest difference in transverse energy, $\Delta E_T$, is chosen
as the most balanced configuration and used to define \alphat. The $M_\text{T}$ for 
the event is insensitive to the clustering and is given by

\begin{equation}
  \label{eq:mt_njet}
  M_\text{T}\, = \,\sqrt{ \left( \sum_{i=1} \Et^{{\rm{j}}_i}
    \right)^2 - \mht^2}.
\end{equation}

\noindent where the sum is over all jets in the event. The sub-leading pseudo-jet energy, 
$E_{\textrm{T}}^{j'_2}$, is given by

\begin{equation}
E_{\textrm{T}}^{j'_2} = \frac{\sum_{i} E_{\textrm{T}}^{j_i} - \Delta E_{\textrm{T}}}{2}.
\end{equation}

The definition of \alphat for any number of jets is therefore

\begin{equation}
  \label{eq:alphat2}
   \alphat = \frac{\sum_{i} E_{\textrm{T}}^{j_i} - \Delta E_{\textrm{T}}}{2\sqrt{\left(\sum_{i} E_{\textrm{T}}^{j_i}\right)^2 - \mht^2}}.
\end{equation}

Most jets in the event contain significant boost such that $E_{\textrm{T}} \sim \pt$. The \alphat 
definition can be approximated in this case by

\begin{equation}
  \label{eq:alphat3}
   \alphat \approx \frac{\scalht - \Delta H_{\textrm{T}}}{2\sqrt{\scalht^2 - \mht^2}}.
\end{equation}

\begin{figure}
\centering
    \includegraphics[width=0.6\textwidth]{./Figures/alphat/alphat_cartoon}
  \caption{\label{fig:alphat_cartoon} The \alphat~variable inputs and values for three types of event: balanced and well measured jets (left), balanced jets with
  a mismeasured jet (middle) and well measured jets recoiling against true~\met (right). The solid cones signify the reconstructed jet momentum while the 
  dashed cones represent the true momentum. The calculation of \alphat~is described in the text.} 
\end{figure}
\noindent where $\Delta H_{\textrm{T}}$ is the difference in \pt~of the pseudo-jets.

To illustrate the mechanism by which the \alphat~variable rejects $\met$ from mismeasured or lost jets,
consider a dijet event as shown in Figure~\ref{fig:alphat_cartoon}. Three possible event topologies are shown: 
a perfectly balanced event (left), a perfectly balanced event with a mismeasured jet (middle) and an event containing true \met~(right).
In a balanced dijet event without mismeasurement, $\Delta E_T = \mht = 0$ and the 
value of \alphat will be 0.5. If one of the jets is undermeasured or overmeasured for an otherwise balanced dijet
event, $\Delta H_{\textrm{T}} = \mht$ and Equation~\ref{eq:alphat3} can be written as 

\begin{equation}
  \label{eq:alphat4}
   \alphat \approx \frac{1}{2}\sqrt{\frac{\scalht - \mht}{\scalht + \mht}} < 0.5.
\end{equation}

Conversely, if an event contains true $\met$ and the jets are recoiling against significant~\met 
(which is not aligned with one of the jets in the event), as shown on the right of 
Figure~\ref{fig:alphat_cartoon}, $\alphat > 0.5$. 

In the general case of two or more jets, for events containing \met~from mismeasurement or 
neutrinos produced in heavy-flavour decays, the values of $\Delta E_{\textrm{T}}$ and \mht~are highly correlated, leading
to values of $\alphat \le 0.5$. This correlation is much weaker in the case of pair-produced, 
R-parity conserving SUSY events, where each decay chain ends in the 
undetected LSP, and for the electroweak backgrounds, allowing $\alphat > 0.5$.

\begin{figure}[!htb]
  \centering
    \includegraphics[width=0.6\textwidth]{./Figures/alphat/alphat_data.pdf}
  \caption{
    The \alphat distribution observed in data compared to simulation. 
    The statistical uncertainties for the QCD multijet and electroweak backgrounds are represented by the hatched areas. 
    The final bin of each distribution contains the overflow events. The events below $\alphat= 0.55$ use unbiased triggers with
    a loose preselection while events above $\alphat = 0.55$ use the signal region triggers and selection.
    }
  \label{fig:alphat-data}
\end{figure}

The \alphat~distribution is shown in Figure~\ref{fig:alphat-data}. The region of $\alphat < 0.5$ is dominated by
QCD multijet events. This contribution sharply drops as \alphat~increases above 0.5. Multijet events with
very rare large stochastic fluctuations in the measured jet energies can lead
to values of \alphat~above 0.5. QCD multijet events may also have \alphat values larger than 0.5 if the 
\pt of one or more jets is sufficiently different from $E_T$, breaking the assumption required for the approximate equality in Equation~\ref{eq:alphat4}. However, this is found to have $< 1\%$ effect on the number of QCD multijet events passing $\alphat > 0.5$.
The \alphat~distribution becomes more sharply peaked with increasing \scalht~in the event as mismeasurements are larger
 for lower jet \pt. Significantly larger values of \alphat~for QCD multijet events can also be caused by 
hot cells or acceptance effects. These are mitigated by the other discriminating variables 
discussed below and dedicated event filters. 

Unlike the QCD processes, the electroweak backgrounds have a long tail in
values of $\alphat$ greater than 0.5. The \alphat~variable allows a powerful discrimination 
between the otherwise dominant QCD background and processes with \met~well
separated from the jets.

\subsection{\bdphi}
\begin{figure}
\centering
    \includegraphics[width=0.6\textwidth]{./Figures/alphat/bdphi_cartoon}
  \caption{\label{fig:bdphi_cartoon} The \bdphi~variable inputs and values for a mismeasured balanced three jet event.
  The third jet minimises the $\Delta \phi_i$ and is used to calculate \bdphi~as described in the text.}
\end{figure}
The \bdphi~variable is an additional topological variable designed to mitigate contamination
from mismeasured events (from reconstruction and instrumental issues) and semileptonic 
heavy-flavour decays. The variable is defined as follows:
\begin{itemize}
\item Each jet in the event is considered in turn as the probe jet, $j_i$.
\item The \mhtvec~is recalculated with the probe jet removed, $\mhtvec^{j_i}$.
\item The azimuthal separation of the probe jet and $\mhtvec^{j_i}$ is calculated, $\Delta \phi_i$.
\item The \bdphi~is calculated as the minimal $\Delta \phi_i$ over all jets in the event,
\begin{equation}
\label{equ:bdphi}
\bdphi = \underset{\forall\, i\, \in \left[1,\nj\right]}{min} \Delta \phi_i.
\end{equation}
\end{itemize}

If the \bdphi~value is close to the jet cone size, the jet which minimizes $\Delta \phi_i$ (\jbdphi)~is likely 
to be mismeasured or to contain~\met~from a leptonic decay of a heavy-flavour quark. The \bdphi~variable is insensitive to
the~\pt of \jbdphi~and rejects events containing a jet whose \pt~is either overmeasured or undermeasured.
The jets in the electroweak background processes and signal models typically recoil against the
\met, implying larger values of \bdphi. The calculation of \bdphi~for a mismeasured balanced 
event containing three jets is shown in Figure~\ref{fig:bdphi_cartoon}. 
In the \alphat~analysis, a threshold of 0.5 is used to reject QCD multijet events.

The \bdphi~variable cannot be defined in the monojet category. For events in this category
containing additional jets with $\pt > 25\,\GeV$, the \bdphi variable is defined 
including these jets. For events with no addition jets with $\pt > 25\,\GeV$,
no threshold on \bdphi is used.

% The \bdphi~variable can be compared to the nominal minimal $\Delta \phi$~between
% the \mht~and the lead four jets in the event used by many hadronic analyses, \dphimhtj.
% The advantages of \bdphi~include sensitivity to mismeasurement of any jet in the 
% event (not just the lead four), sensitivity to both undermeasrement and overmeasurement
% of jet energies (\dphimhtj~does not provide rejection of events with an overmeasured jet)
% and insensitivity to the \pt~of the jet that is mismeasured. Given these advantages, 
% the \bdphi~variable can  an order of magnitude better rejection of QCD for the 
% same signal efficiency for a SUSY model containing significant \met~and hadronic activity.

\begin{figure}[!htb]
  \centering
    \includegraphics[width=0.6\textwidth]{./Figures/alphat/bdphi_data.pdf}
  \caption{
    The \bdphi~distribution observed in data compared to simulation for a selection $\scalht > 800\,\GeV$.
    The statistical uncertainties for the multijet and SM expectations are represented by the hatched areas. 
    }
  \label{fig:bdphi-data}
\end{figure}

The distribution of \bdphi~in data and simulation for $\scalht > 800\,\GeV$ is shown in Figure~\ref{fig:bdphi-data}.
The QCD multijet background is seen to decrease by around five orders of magnitude as \bdphi increases
to 0.5. In Figure~\ref{fig:bdphi-data} the electroweak backgrounds are
shown to have long tails up to $\bdphi = \pi$.

\subsection{\mhtmet}
\label{sec:mhtmet}
The \mhtmet~cleaning cut is designed to reduce the contamination from balanced events
which contain significant \mht~due to acceptance effects from kinematic and 
pseudorapidity thresholds, severe jet mismeasurement and particles not 
clustered into jets. The PF \met is more robust against acceptance effects as all PF candidates are
included in its computation. The contamination due to such effects may therefore be mitigated 
using a maximal threshold on the ratio of the \mht~to the \met.

\subsection{Forward jet veto}
\label{sec:fwd_jet_veto}
The $\mht$ variable is made using jets with $\etaabs < 2.4$. Jets in the forward pseudorapidity 
region may introduce $\mht$. A veto on any jet in the forward region with $p_T > 40\,\GeV$
is used to reject such events. 

\section{Preselection}
\label{sec:presel}
The objects described in Section~\ref{sec:phys-obj} are used to define a common kinematic 
preselection for the signal and control regions used in the $\alphat$ analysis. 
The kinematic selections as well as event filters and vetoes targeting
known instrumental, reconstruction and beam effects are described in this section.

\subsection{Kinematic selections}
The minimal requirements on the energy sums for both the signal and control regions are
$\scalht > 200\,\GeV$ and $\mht > 130\,\GeV$. In addition, at least one jet satisfying the 
requirements detailed in Table~\ref{tab:kine-sel} is required. These selections are 
designed by considering trigger efficiency constraints while maximising acceptance for
signal models with small mass splittings.

\subsection{Event filters}
\label{sec:event_filters}
Event filters reject events contaminated by detector or reconstruction effects
which can cause significant~\met. Such effects may not be known when the 
background simulations are produced or can be difficult to simulate. Therefore,
dedicated filters are used to remove such events from the dataset. These `\met filters' 
are:

\begin{itemize}
\item The HBHE noise and isolation filters that rejects energy spikes in the HCAL caused by noise from sources 
including direct particle interactions with HCAL instrumentation.
\item The EE bad supercluster filter that removes events containing \TeV~scale energy spikes from anomalous
pulses in the EE.
\item The CSC beam halo filter that uses the CSC to measure halo muons and reject events containing 
significant contamination from the beam halo.
\item The ECAL trigger primitive filter that rejects events where significant energy is deposited in dead cell regions.
\item The PV filter that requires at least one well reconstructed vertex, which ensures collisions have occurred (rather than an event triggered and reconstructed from detector noise).
\item The bad muon and bad charged hadron filters that reject events containing poorly reconstructed muons reconstructed as 
a muon or PF charged hadron, respectively.
\end{itemize}

\begin{figure}
\centering
    \includegraphics[width=0.6\textwidth]{./Figures/alphat/met_clean}
  \caption{\label{fig:met_filter} The $\met$~filters remove the long tail in the reconstructed \met~caused by detector, reconstruction
  and beam effects in the data~\cite{met_fig}.} 
\end{figure}
The performance of the \met~filters for the reconstructed \met~in the data is shown in Figure~\ref{fig:met_filter}. The long tail caused by
the effects in data discussed above is mitigated by the filters. The typical signal efficiency for these filters
is $> 99\%$. In addition to the \met~filters, additional vetoes are used by 
the \alphat~analysis to reject events affected by misreconstruction and residual beam halo, as follows:
\begin{itemize}
\item A leading jet charged hadron energy fraction (CHF) veto that rejects events 
contaminated by beam halo energy deposits which pass the CSC beam halo filter. The lead jet 
in the event is required to satisfy CHF > 0.1.
\item The odd jet veto that rejects any event containing a jet which fails the 
requirements in Table~\ref{tab:loose-jet-id}. Such events are often caused by noise or misreconstruction.
\item The forward jet veto, described in Section~\ref{sec:fwd_jet_veto}, rejects events containing jets which fall out of acceptance but
are reconstructed in the forward region. 
\item The $\mhtmet < 1.25$ requirement, described in Section~\ref{sec:mhtmet}, rejects events containing 
\mht~due to energy present in the event that is not clustered into jets. 
\end{itemize}
%%MAYBE ADD PLOTS HERE TO MOTIVATE?

\section{Analysis selections}
The common preselections are applied for all events used by the \alphat~analysis. The events 
passing these selections may then fall into the signal region or the hadronic,
\gj, \mj~or \mmj~control region. These regions are exclusive so that no
event may be double counted; the selections for each are summarised below.

\subsection{Hadronic signal region}
\label{sec:had_sr}
Events in the signal region are categorised (see Section~\ref{sec:cat}) and used
to search for signatures of new physics (see Chapter~\ref{cha:statisticalResults}). 
Selections are designed to suppress the QCD multijet and 
electroweak backgrounds while maintaining acceptance to 
hadronic SUSY models containing significant \met.

The QCD multijet background is suppressed using thresholds on 
the \alphat, \bdphi~and \mhtmet~variables and event filters.
An \scalht~dependent threshold is used to account for the broadening \alphat~distribution
for lower \scalht values. These thresholds are also chosen to be efficient given
the trigger requirements (see Section~\ref{sec:ana-trigger}) and are shown in Table~\ref{tab:alphat-thresholds}. 
The highest threshold of 0.65 is used for $200\,\GeV < \scalht < 300\,\GeV$ while there is no \alphat
requirement for $\scalht > 800\,\GeV$, for which region a pure \scalht~trigger is used.
At least two jets are required to define~\alphat and therefore no threshold is used
in events containing only one jet (monojet).

\begin{table}[h!]
  \caption{\alphat thresholds versus
    lower bound of \scalht bin in units of $\GeV$. For all \scalht bins satisfying $\scalht >
    800\,\GeV$, no \alphat cut is applied. No \alphat requirement is
    imposed in the monojet bins.}
  \label{tab:alphat-thresholds}
  \centering
  \footnotesize
  \begin{tabular}{ lccccccccc }
    \hline
    \hline
    \scalht            & 200       & 250       & 300       & 350       & 400       & 500       & 600 &  $>$800    \\
    \hline                                                                                     
    \alphat threshold  & 0.65      & 0.60      & 0.55      & 0.53      & 0.52      & 0.52      & 0.52 & --    \\
    \hline
    \hline
  \end{tabular}
\end{table}

A \bdphi~threshold of 0.5 is used for all events passing signal region selection.
This value is motivated by the jet cone size of 0.4 and desired level of 
QCD suppression. For monojet events, the \bdphi variable is defined using all jets with 
$\pt > 25\,\GeV$. If there are no such jets, no threshold on \bdphi is used.

The values of the thresholds on \alphat, \bdphi~and \mhtmet~provide the desired 
level of QCD multijet contamination, measured in data, of $< 10\%$
(see Section~\ref{sec:qcd-pred}). This level of QCD rejection is unique for the 
\alphat~analysis among hadronic SUSY searches. Given the difficulties in predicting
residual QCD backgrounds (see Section~\ref{sec:qcd-pred}), the \alphat strategy can provide a robust 
confirmation of a potential discovery.
%
% \subsubsection{Electroweak background suppression}
%
% The requirement of a purely hadronic final state suppresses 
% electroweak backgrounds containing \met~from final states including
% leptons (including \wl~and \tlb). The isolated track veto 
% additionally rejects single prong decays of the $\tau$. 

\subsection{Control regions}
\label{sec:cr-sel}
The control regions are used to predict both the residual QCD multijet
and electroweak backgrounds. The predictions are made using the 
`transfer factor' (TF) method, as described in Section~\ref{sec:tf-pred}.
Each of the control regions is enriched 
in the background sample(s) predicted by that control region or a related 
process. The selections used are described in this section and are designed to 
be as close to the signal region selections as possible 
while ensuring a large sample of events enriched in the relevant process. 

\subsubsection{Hadronic control regions}
The hadronic control regions are used to determine the level of QCD contamination
remaining in the signal region (see Section~\ref{sec:qcd-pred}). These are defined by inverting the selection on the 
discriminating variables described in Section~\ref{sec:important-variables} to give a data sample 
dominated by events from QCD processes. Any event containing a reconstructed photon or lepton is vetoed.

\subsubsection{Photon control region}
The \gj~control region is used to predict the \znunu~background in the signal region.
One photon satisfying the requirements in Table~\ref{tab:kine-sel} is required in the 
event and events containing leptons are vetoed. The energy sums are then calculated without this photon. 
To reduce the reliance on simulation, all signal region selections are made except for 
those on the \bdphi~variable. These requirements are removed to enhance 
the statistics in the control region sample 
for predicting the \znunu~background. Due to trigger considerations, the \gj control region
is used for $\scalht > 400\,\GeV$ only. To ensure the photon is not affected by hadronic
activity, any events with a jet satisfying $\Delta R(j,\gamma) < 1$ are vetoed. Finally,
the $\mht > 130\,\GeV$ threshold in both signal and control helps ensure the bosons are 
sufficiently boosted such that the Z boson mass does not bias the 
prediction of \znunu using the \gj control region.

\subsubsection{Muon control regions}
The muons used for the muon control regions are defined as detailed in
Table~\ref{tab:kine-sel}. Energy sums are defined without including the muon 
and all selections except for those on the \bdphi~and \alphat~variables are used.
Events are vetoed if $\Delta R(j,\mu) < 0.5$ for all jets and muons in the event to
ensure no hadronic activity affects the muon(s). Events are also rejected if they 
contain an electron or photon.

The \mj~control region is used to predict the lost lepton background from 
a sample rich in \wj~and \ttbar~processes. A single muon is required and a selection is used on the transverse mass variable,
$\mt(\mu,\met) \equiv \sqrt{2E^{\mu}_{T}\met(1-\cos(\Delta\phi_{\mu,\met}))}$,
of $\mt(\mu,\met) < 125\,\GeV$ to enrich the sample with the \wj~background 
and to reduce potential contamination from SUSY signal models.

The \mmj~control region is used along with the \gj~control region to predict the \znunu~background
using a sample rich in the \zmmj~process. Exactly two oppositely charged muons are required with
an invariant mass within 25 \GeV~of the mass of the Z boson, $|M_{\mu_1,\,\mu_2} - m_Z| < 25\,\GeV$.

Electron control regions can provide complementary statistical power to the muon control regions. 
However, they are not used in the \alphat analysis due to the significant increase in 
the rate of fakes from QCD multijet processes compared to the muon control regions. 


\section{Categorisation}
\label{sec:cat}
In order to maintain sensitivity to a wide range of signal models, events are categorised using
several discriminating variables. This categorisation allows sensitivity to a wide range of 
hadronic signal models. 

The \alphat~analysis uses a categorisation in \njet, \nb~and \scalht~for both 
the signal and control regions. Each signal region category is therefore predicted using an identical 
selection in these variables for the control region. In determining the categorisation, 
sufficient control region statistics are required to predict each \njet, \nb~and \scalht~bin. 
The signal region is further categorised according to~\mht. The categorisations and motivations are detailed below.

\subsection{Categorisation in \njet}

The \alphat~analysis categorises events into three topologies, `monojet', `asymmetric' 
and `symmetric', using selections on the second jet~\pt. For all topologies, the leading
jet must satisfy $\pt^{j_1} > 100 \,\GeV$.  

The monojet topology vetoes events containing a second 
jet with $\pt^{j_2} > 40 \,\GeV$ and is designed to target compressed SUSY models. For such models the 
LSP carries the majority of the energy from the heavy sparticle decay, meaning any jets produced 
in the decay are soft (small values of \pt). These events may fall into the monojet category if
one of the incoming/outgoing partons radiates a jet. The monojet topology rejects backgrounds from 
\ttbar which tend to have high \njet multiplicities. 

The asymmetric topology selects events containing a second jet with $40 \,\GeV < \pt^{j_2} < 100 \,\GeV$.
This category targets compressed SUSY models whose decay products are soft but can be reconstructed.
In the presence of a boosted ISR/FSR such an event may have significant imbalance between
the \pt of the ISR/FSR jet and the soft jets from the sparticle decay.

Finally, the symmetric topology requires $\pt^{j_2} > 100 \,\GeV$ and targets more
generic supersymmetric models with large mass splittings 
for which the decay products carry significant energies.

The jet multiplicity varies widely depending on the signal model and background process. Supersymmetric models of light flavoured squark production and compressed models tend to produce 
a small number of jets with significant energy. Conversely, uncompressed models of gluino or top squark production
have larger jet multiplicities. The backgrounds are also largely \njet~dependent as 
\ttbar~decays typically produce high multiplicities while decays of vector boson backgrounds
produce low multiplicities. The events with a symmetric topology are therefore categorised according
to $\njet = 2,3,4$ and $\njet \ge5$.

\subsection{Categorisation in \nb}

A categorisation on \nb~improves sensitivity to models containing direct and indirect
decays via heavy-flavour squarks. Such models are favoured as they naturally
predict a light Higgs mass (see Chapter~\ref{cha:theory}) and can produce
final states including two to four bottom quarks, depending on the decay mode.
Events are categorised as $\nb = 0,1,2$ and $\nb \ge3$. The dominant background for events 
with more than one b jet comes from the \ttbar process. The misidentification of 
jets from light-flavoured quarks as b jets increases the background rate for a high $\nb$ selection.

\subsection{Categorisation in \scalht}

The \scalht~produced in an event provides a measure of the mass scale of 
new physics models as the energy provided to the visible decay products is
dependent on the mass splitting between the heavy sparticle and LSP. 
The finest control and signal region binning in \scalht~is shown in Table~\ref{tab:alphat-thresholds}. 
These are merged from above depending on the requirement of at least one control region
event observed in data per \scalht~region to allow a meaningful prediction.

% \begin{table}[h!]
%   \caption{The definition of the lower boundaries of the bins for the hadronic signal region
%  and control regions. The next lower boundary defines the upper boundary for each bin.}
%   \label{tab:ht-binning}
%   \centering
%   \footnotesize
%   \begin{tabular}{ lcccccccccc }
%     \hline
%     \hline
%     \scalht~Binning (\GeV)           & 200      & 300       & 350       & 400       & 500       & 600       & 750 & 900 & 1050 & $\ge 1200$  \\
%     % \hline                                                                                     
%     % Signal region  \scalht (\GeV)         & 200         & --      & --      & 400  & --        & 600 & -- & 900  & -- & $\ge 1200$  \\
%     \hline
%     \hline
%   \end{tabular}
% \end{table}

\subsection{Categorisation in \mht}
\label{sec:cat-mht}
The categorisation in \mht~was introduced for the first analysis on
Run 2 data. The \mht~modelling in the signal region is taken from 
simulation and a data-driven approach is utilised to validate 
this modelling using the data in the control regions as described 
in Section~\ref{sec:syst-on-shape}.

The \mht~variable is dependent on the energy carried by
the invisible decay product of any BSM physics model (such as the
LSP in SUSY models). For compressed
models, the categorisation in both \scalht~and \mht provides additional 
sensitivity as, unlike the SM backgrounds, such models tend to 
have \mht~values near \scalht. 

In determining the categorisation, a maximum statistical uncertainty from the 
simulated events of 50\% (corresponding to four unweighted events) 
is required in each bin of the template in order to ensure a statistically meaningful prediction. 
In addition, a minimum bin width constraint of 50 \GeV~is applied in order to reduce bin-by-bin migration 
due to the finite \mht resolution.

\subsection{Summary}

In Table~\ref{tab:binning-summary}, the lower bounds of the first and final \scalht
bins for each jet category are shown. The categories are further binned in \mht
according to the metric defined in Section~\ref{sec:cat-mht}.

\begin{table}[htb!]
  \caption{Summary of the lower bounds of the first and final bins
    in \scalht (the latter in parentheses) in units of \GeV~as a function of \njet and
    \nb. Intermediate \scalht bins are taken from the values shown in Table~\ref{tab:alphat-thresholds}.
  \label{tab:binning-summary}}
  \centering
  \footnotesize
  \begin{tabular}{ lcccc }
    \hline
%    \njet                   & \multicolumn{4}{c}{\nb}                                           \\
%    \cline{2-5}
%                            & 0         & 1         & 2         & $\geq$3                       \\
    $\njet \backslash\, \nb$ & 0         & 1         & 2         & $\geq$3                       \\
    \hline
    \multicolumn{5}{l}{\bf Monojet}                                                              \\
    1                        & 200 (600) & 200 (500) & -     & -                         \\
    \multicolumn{5}{l}{\bf Asymmetric}                                                           \\
    2                        & 200 (600) & 200 (500) & 200 (400) & -                         \\
    3                        & 200 (600) & 200 (600) & 200 (500) & 200 (300)                     \\
    4                        & 200 (600) & 200 (600) & 200 (600) & 250 (400)                     \\
    $\geq$5                  & 250 (600) & 250 (600) & 250 (600) & 300 (500)                     \\
    \multicolumn{5}{l}{\bf Symmetric}                                                            \\
    2                        & 200 (800) & 200 (800) & 200 (600) & -                         \\
    3                        & 200 (800) & 250 (800) & 250 (800) & \phantom{0}-\phantom{0} (250) \\
    4                        & 300 (800) & 300 (800) & 300 (800) & 300 (800)                     \\
    $\geq$5                  & 350 (800) & 350 (800) & 350 (800) & 350 (800)                     \\
    \hline
  \end{tabular}
\end{table}

\section{Trigger strategy}
\label{sec:ana-trigger}

\subsection{Hadronic signal and control regions}

The trigger strategy is designed to maximise acceptance for an inclusive set of new physics 
models which may be produced at a low energy scale. A suite of dedicated cross triggers 
is used at the HLT to achieve efficiencies near 100\% for selections on
\scalht~and \mht~as low as 200 and 130 \GeV,~respectively. 

The HLT triggers used for the signal region are summarised in Table~\ref{tab:trigger} as well as the L1 triggers
used to seed these paths. In order to reduce the rate sufficiently to allow PF reconstruction at the HLT,
a prefilter is required on calorimeter quantities. An event in any category may pass any of the trigger
paths; however, each is designed to maximise the efficiency for a particular set of categories.

For the monojet topology, the highest efficiency is achieved with the HLT
$\mht-\met$ cross trigger using thresholds of 90 \GeV~on both quantities. For the symmetric and asymmetric
categories, dedicated seeds are used that select events based on the \alphat, \scalht and the 
average \pt of the leading two jets ($p^{(j_1,\,j_2)}_{\text{T}}$). A selection of
$p^{(j_1,\,j_2)}_{\text{T}} > 90\,\GeV$ was found to optimise the efficiency across topologies given the rate requirements. The resultant reduction in efficiency for 
asymmetric categories is recovered by the use of the $\mht-\met$ cross trigger in addition to the dedicated cross triggers.
Finally, for bins with $\scalht > 800 \,\GeV$, where there is no \alphat threshold, 
an HLT seed with a threshold of only $\scalht > 800\,\GeV$ provides efficiency.


\begin{table}[h!]
\caption{Trigger thresholds of the Level-1 and final PF-trigger decision for
 the HLT paths for the hadronic signal region. Except for those on~\alphat, all thresholds are in \GeV.}
\footnotesize
\centering
\begin{tabular}{c|cccc} 
\hline
\hline
L1 seed & HLT calo-prefilter & HLT PF-filter                                                \\
($\scalht$,$\met$) & ($\scalht$, $\alphat$, $\pt^{\rm \left<j1,j2\right>}$, \met) & ($\scalht$, $\alphat$, $\pt^{\rm \left<j1,j2\right>}$, \met) \\ %& (Hz) \\[0.7 ex] 
\hline
240, 70 & 150, 0.540, 70, - & 200, 0.570, 90, - \\ %& \\ % 11.0 $\pm$ 3.0 \\
240, 70 & 200, 0.535, 70, - & 250, 0.550, 90, - \\ %& \\ % 8.5  $\pm$ 3.0 \\
240, 70 & 250, 0.525, 70, - & 300, 0.530, 90, - \\ %& \\ % 9.5  $\pm$ 3.0 \\
240, 70 & 300, 0.520, 70, - & 350, 0.520, 90, - \\ %& \\ % 10.0 $\pm$ 3.0 \\
240, 70 & 370, 0.510, 70, - & 400, 0.510, 90, - \\ %& \\ % 13.5 $\pm$ 3.5 \\ \\ %\hline \\ %
240, -  & 650, -, -, -      & 800, -, -, -   \\ %& \\ % 13.5 $\pm$ 3.5 \\ \\ %\hline \\ %
 -, 70  &   -, -, -, 65   &  -, -, -, 90    \\
%% L1sL1ETM70ORETM60ORETM50
%% hltMET, MHT 65
\hline
\hline
\end{tabular}
\label{tab:trigger}
\end{table}

The trigger efficiencies for the signal region are measured using the efficiency for 
data selected by independent muon and electron `reference triggers' to pass signal 
region selection. The central value of the trigger efficiency
is taken from the measurement using the electron reference trigger. The efficiency in \mht~
bins for different \scalht~ranges is shown in Figure~\ref{fig:alphat_turnons}. Given the selection of $\mht > 200\,\GeV$,
an efficiency of $>90\%$ is achieved across the signal region. 

Biases may be introduced in the trigger efficiency measurement due to contamination in the computation
of event variables from the reference object and different treatments between trigger and offline reconstructions.
The difference between the measurement of the efficiencies using the muon and electron reference 
triggers is used to probe this bias and propagated as a systematic uncertainty. 

\begin{figure}[h!]
  \begin{center}
    \subfloat[$400 < \scalht < 600$, symmetric topology]{\includegraphics[width=0.45\textwidth]{Figures/alphat/Trigger/HLT_AlphaTMonoAll_MoM_400to600_mht}}~~
    \subfloat[$400 < \scalht < 600$, asymmetric topology]{\includegraphics[width=0.45\textwidth]{Figures/alphat/Trigger/HLT_AlphaTMonoAll_MoM_Asym_400to600_mht}} \\
    \subfloat[$200 < \scalht$, monojet topology]{\includegraphics[width=0.45\textwidth]{Figures/alphat/Trigger/HLT_MonoAll_MoM_Mono_MHT0_ht}}
    \caption{Signal trigger efficiency in the \mht dimension measured with a muon sample for a representative category for each topology.}
    \label{fig:alphat_turnons}
  \end{center}
\end{figure}
%%%MUST ADD TRIGGER EFFICIENCY!!!!

\subsection{Electroweak control regions}

The control regions use a trigger requirement on the relevant object in that control region.
The \mj~and \mmj~regions are selected using a threshold of 22 \GeV~on the muon \pt at the HLT.
The efficiency is measured in data using the tag and probe method~\cite{MuonReco}. These efficiencies are 
applied as appropriate per muon \pt and $\eta$. 

The \gj control region is selected using events passing a threshold at the HLT of photon $\pt > 175 \,\GeV$ 
or \scalht > 800\GeV. The efficiency is measured in data using events passing the HLT $\scalht > 800\,\GeV$ 
requirement that satisfy the \gj control region requirements. These efficiencies are shown in
Figure~\ref{fig:photon_turnons_photonPt} and are applied per photon \pt.

\begin{figure}[h!]
  \begin{center}
    \includegraphics[width=0.6\textwidth]{Figures/alphat/Trigger/Photon/HLT_PhotonECALHT800_MoM_all_all_gammapt}
    \caption{Trigger efficiency as a function of photon \pt for an inclusive selection.}
    \label{fig:photon_turnons_photonPt}
  \end{center}
\end{figure}

\section{Summary}

The selections described in this section provide a powerful rejection of the dominant QCD multijet background as well as significantly reducing the electroweak backgrounds. 
The events falling into the signal region are finely categorised in order to provide inclusive sensitivity to a wide range of BSM models. 
To extract a possible signal contribution, the residual backgrounds must be robustly estimated, as described in Chapter~\ref{cha:backgroundPrediction}.

